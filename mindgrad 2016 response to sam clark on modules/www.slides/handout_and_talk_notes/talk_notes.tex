%!TEX TS-program = xelatex
%!TEX encoding = UTF-8 Unicode

%\def \papersize {a5paper}
\def \papersize {a4paper}
%\def \papersize {letterpaper}

%\documentclass[14pt,\papersize]{extarticle}
\documentclass[12pt,\papersize]{extarticle}
% extarticle is like article but can handle 8pt, 9pt, 10pt, 11pt, 12pt, 14pt, 17pt, and 20pt text

\def \ititle {Joint Action: Talk Notes}
\def \isubtitle {Lecture 01}
%comment some of the following out depending on whether anonymous
\def \iauthor {Stephen A.\ Butterfill}
\def \iemail{s.butterfill@warwick.ac.uk% \& corrado.sinigaglia@unimi.it
}
\input{$HOME/Documents/submissions/preamble_steve_lecture_notes}

%no indent, space between paragraphs
\usepackage{parskip}


%for e reader version: small margins
% (remove all for paper!)
%\geometry{headsep=2em} %keep running header away from text
%\geometry{footskip=1.5cm} %keep page numbers away from text
%\geometry{top=1cm} %increase to 3.5 if use header
%\geometry{bottom=2cm} %increase to 3.5 if use header
%\geometry{left=1cm} %increase to 3.5 if use header
%\geometry{right=1cm} %increase to 3.5 if use header

% disables chapter, section and subsection numbering
\setcounter{secnumdepth}{-1}

%avoid overhang
\tolerance=5000

%\setromanfont[Mapping=tex-text]{Sabon LT Std}


%for putting citations into main text (for reading):
% use bibentry command
% nb this doesn’t work with mynewapa style; use apalike for \bibliographystyle
% nb2: use \nobibliography to introduce the readings
\usepackage{bibentry}

%screws up word count for some reason:
%\bibliographystyle{$HOME/Documents/submissions/mynewapa}
\bibliographystyle{apalike}


\begin{document}



\setlength\footnotesep{1em}




%---------------
%--- start paste




What is a module?
Opponents of the claim that there are modules often suppose that we can explain
what it is for something to be a module by giving a list of features.
They take theoretical committment to modularity to be merely committment to the existence
of a systems which exhibit
informational encapsulation,
domain specificity,
limited access,
and the rest.

But Sam has argued that these features are merely superficial characteristics
of modules.

\subsection{slide-4}
On his view,
the fact that a system is a module explains why it exhibits these features.
So of course being a module cannot simply consist in bearing these features.

To repeat, modularity is something that explains these properties rather
than something that merely consists in them.

Going further, Sam suggests that being modular is a matter of your processes being
rule-govered computational processes of a certain type, and of the representations you
operate on having a distinctive representational format.

It is these facts about the distinctively modular processes and representations
that explain why modules evidence features like information encapsulation or domain
specificity.

It follows that
what matters in determining whether something is module
is not the absolute degree of encapsulation (say), but rather
whether differences in degree of encapsulation (say)  indicate
differences in the nature of the underlying processes and representational formats.

A threat that Sam is aware of: opponents of modularity will retort that Sam has
raised the bar for objections so high that nothing could show
there are no modules!

[As Sam writes in the paper,
‘Concerning how Fodorians should understand the natural kindness of modular input systems, my
suggestion is just that modular systems might be distinguished in the human mind by an underlying
type of computational process that governs their operations: i.e. that the essence of the modular
kind might be a certain type of computational processing that is found in some, and only some,
cognitive systems—the modular ones—and that entails and explains the manifestation of the
superficial properties that these systems are seen to display to a striking degree.’]

\subsection{slide-5}
Sam’s retort (in the paper) is to emphasise that his view makes some predictions ...

‘predictions ... whenever we find a module we will find that the content it processes is like that processed by other modules ... and unlike that processed by non-modules’ with respect to its representational format.

\subsection{slide-6}
In what follows I want to focus on this notion of representational format.

\subsection{slide-7}
It’s a familiar idea that things---sounds, say---can be specified using
different representational formats.

Differences in format matter because they affect performance: some things that
are difficult or impossible in one format (changing pitch, say) can be
possible and perhaps even easy in another representational format.

\subsection{slide-8}
I take it that nonmodular processes involve representations with different formats.
In producing music, say, we might use artifactual representations with varying
representational formats.
And it seems reasonable to suppose that the mental representations involved might similarly
have a variety of different representational formats.

Or think about finding your way here to MindGrad, which could involve
combining use of written instructions with a paper map.
Again, it seems reasonable to think that performing this task may also involve
mental representations with various formats.

[This is the beginnings of an objection to something Sam writes:
‘critics of such a view seem required to hold the opposite: they must hold that the form and
structure of the representations will remain constant across these systems, at least from the
point of view of the systems doing the computing.’]

\subsection{slide-9}
How are things with modules?  Is there a single format common to all the representations
in modules, or can representational format differ from one module to the next?

Sam’s view is most straightforwardly interpreted as denying this possibility.
The claim is that modularity has an essence, and that essence is specified in part by
a representational format that distinguishes modular from nonmodular processes.
It would be simplest to develop this view on the assumption that there is
one kind of rule-goverened computational process and one kind of
representational format common to all modules.

Sam himself appears to accept this view:
‘the essence of the modular kind might be a certain type of computational processing that is found
in ... only ... the modular  ... and that entails and explains the
manifestation of the superficial properties that these systems are seen to display to a striking
degree.’

\subsection{slide-10}
But it seems quite plausible that different modules will involve
different representational formats.  Consider the modules that Sam mentions:
vision, speech and navigation.

Vision presumably deals with the arrangement of surfaces and objects in space,
whereas navigation is about landmarks and routes;
and speech concerns acoustic and bodily events.

It is tempting to suppose that dealing with these things efficiently may require
representations with different formats, and so that the
representations in three modules will differ in format.

Suppose this is right, that the underlying processes differ
significantly in kind.

Recall that it is the nature of their processes and representational formats which explain
why modules evince features such as information encapsulation.
This may well mean that these three modules---speech perception,
vision and navigation---evince charcateristics associated with modularity such as encapsulation
and limited accessibilities to varying degrees.

What follows from this possibility?

[Sam also writes that
‘critics must convincingly show that different modules manifest their distinctive properties in
quite different ways such that these differences must be explained by the underlying properties or
mechanisms of the systems themselves’]

\subsection{slide-12}
An immediate consequence, given Sam’s view, is that there would be no one
such thing as modularity.
After all, the unity of the modules depends on their having a single essence.
But if the essence of modularity is specified by the nature of the processes
and the format of the representations, then the idea that different putatively
modular systems involve different kinds of process and format seems to imply that
they lack a single essence.

This might seem like an objection to Sam’s view, but I’m not sure it is.
I have the sense that I am merely taking the view that Sam has presented
and going one step further.

After all, once we have the distinction between format and kind of process
as a theoretical tool, why do we need there to be a single essence which all
modules share?

But I’m not sure.  Maybe this is a step too far.
Let’s see what Sam says.
Here are my three questions for him.


        



%--- end paste
%---------------






\bibliography{$HOME/endnote/phd_biblio}



\end{document}
