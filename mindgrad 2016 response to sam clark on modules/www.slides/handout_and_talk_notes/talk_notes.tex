%!TEX TS-program = xelatex
%!TEX encoding = UTF-8 Unicode

%\def \papersize {a5paper}
\def \papersize {a4paper}
%\def \papersize {letterpaper}

%\documentclass[14pt,\papersize]{extarticle}
\documentclass[12pt,\papersize]{extarticle}
% extarticle is like article but can handle 8pt, 9pt, 10pt, 11pt, 12pt, 14pt, 17pt, and 20pt text

\def \ititle {Joint Action: Talk Notes}
\def \isubtitle {Lecture 01}
%comment some of the following out depending on whether anonymous
\def \iauthor {Stephen A.\ Butterfill}
\def \iemail{s.butterfill@warwick.ac.uk% \& corrado.sinigaglia@unimi.it
}
\input{$HOME/Documents/submissions/preamble_steve_lecture_notes}

%no indent, space between paragraphs
\usepackage{parskip}


%for e reader version: small margins
% (remove all for paper!)
%\geometry{headsep=2em} %keep running header away from text
%\geometry{footskip=1.5cm} %keep page numbers away from text
%\geometry{top=1cm} %increase to 3.5 if use header
%\geometry{bottom=2cm} %increase to 3.5 if use header
%\geometry{left=1cm} %increase to 3.5 if use header
%\geometry{right=1cm} %increase to 3.5 if use header

% disables chapter, section and subsection numbering
\setcounter{secnumdepth}{-1}

%avoid overhang
\tolerance=5000

%\setromanfont[Mapping=tex-text]{Sabon LT Std}


%for putting citations into main text (for reading):
% use bibentry command
% nb this doesn’t work with mynewapa style; use apalike for \bibliographystyle
% nb2: use \nobibliography to introduce the readings
\usepackage{bibentry}

%screws up word count for some reason:
%\bibliographystyle{$HOME/Documents/submissions/mynewapa}
\bibliographystyle{apalike}


\begin{document}



\setlength\footnotesep{1em}




%---------------
%--- start paste



What do we experience of action when we act together?

This question is much to complex to be taken whole, at least by me.
To make it more manageable, I want to focus on

\subsection{slide-4}
very-small scale joint actions
such as our (i) clinking glasses together, (ii) passing a mug between our hands,
or (iii) playing a piano chord together.

In performing such very small scale joint actions, what do we expeirence of action?

My aim in posing this question is to get at elements of experience that matter
for understanding empathy in and with groups.
I think empathy in groups often involves, and perhaps even sometimes depends on,
experiences associated with very small-scale joint actions.

We clink glasses in raising a toast, our eyes meet and that feeling of dismay
at the bride’s obviously terrible choice of wife becomes a shared feeling.
Although the target of this workshop is probably the shared feeling, I want to focus
on the experiences associated with merely clinking glasses.
Not because I think these experiences are more important, but because I think
they’re interesting in their own right, they raise some hard questions
and they matter for shared feelings.

You pass me a mug and I take it.
But as I grasp it, you keep hold of it for just a moment too long.
Surprised, I look to your eyes.
This is your moment to express the pain you are feeling.
But my focus in this talk isn’t the pain we end up sharing: it’s the
experience of acting together in passing the mug that you
skilfully interrupt in order to create a moment between us.
What do we experience in performing a very small scale joint action like
passing a mug between us?

\subsection{slide-5}
I want to start by forgetting about joint action and asking a question about
ordinary, individual action.

\subsection{slide-6}
What can we experience
when we encounter
very small-scale purposive actions?

\subsection{slide-7}
By ‘very small scale actions’ I mean things like
grasping [instrumental]
producing an phonemic gesture in speaking [communicative]
or producing a bodily expression of emotion, as in frowning.

\subsection{slide-8}
I take all of these very small scale actions to be goal-directed.
To illustrate, consider producing a phonemic gesture.
This involves a complex coordinated movement of lips, larynx, tounge and velum.
Which movements are required to produce a particular phonemic gesture varies
significantly depending on the context and situation. It is also possible to
fail to articulate a phoneme.
So I take the production of a phonemic gesture to be a purposive action:
the goal is the articulation of a particular phoneme.

Now contrast two views ...

\subsection{slide-9}
The \emph{Indirect Hypothesis}: experiences revelatory of action are all experiences of bodily
configurations, of joint displacements and of effects characteristic of particular actions.

\subsection{slide-10}
The \emph{Direct Hypothesis}: some experiences revelatory of action are experiences of actions as
directed to particular outcomes.

So in observing action we experience not only bodily configurations,
joint displacements, sounds and the rest but also goal-directed actions.

\subsection{slide-11}
To find evidence that might discriminate these views, we first need to
consider the psychological mechanisms that underpin the performance of
very small scale actions \textbf{and that make them the actions they are} ...

\subsection{slide-12}
Let me mention some almost uncontroversial facts about motor representations and
their action-coordinating role.

\subsection{slide-13}
Suppose you are a cook who needs to take an egg from its box, crack it and put it (except for the
shell) into a bowl ready for beating into a carbonara sauce.
Even for such mundane, routine actions, the constraints on adequate performance can vary
significantly depending on subtle variations in context. For example, the position of a hot pan
may require altering the trajectory along which the egg is transported, or time pressures may mean
that the action must be performed unusually swiftly on this occasion.
Further, many of the constraints on performance involve relations between actions occurring at
different times.
To illustrate, how tightly you need to grip the egg now depends, among other things, on the forces
to which you will subject the egg in lifting it later.
It turns out that people reliably grip objects such as eggs just tightly enough across a range of
conditions in which the optimal tightness of grip varies.
This indicates (along with much other evidence) that information about the cook’s anticipated
future hand and arm movements appropriately influences how tightly she initially grips the egg
(compare \citealp{kawato:1999_internal}).
This anticipatory control of grasp,
like several other features of action performance (\citealp[see][chapter 1]{rosenbaum:2010_human} for more examples),
is not plausibly a consequence of mindless physiology, nor of intention and practical reasoning.
This is one reason for postulating motor representations, which characteristically play a role in
coordinating sequences of very small scale actions such as grasping an egg in order to lift it.

The scale of an actual action can be defined in terms of means-end relations.
Given two actions which are related as means to ends by the processes and representations
involved in their performance, the first is smaller in scale than the second just if the
first is a means to the second.  Generalising, we associate the scale of an actual action
with the depth of the hierarchy of outcomes that are related to it by the transitive closure
of the means-ends relation.  Then, generalising further, a repeatable action (something that
different agents might do independently on several occasions) is associated with a scale
characteristic of the things people do when they perform that action.  Given that actions
such as cooking a meal or painting a house count as small-scale actions, actions such as
grasping an egg or dipping a brush into a can of paint are very-small scale.  Note that we
do not stipulate a tight link between the very small scale and the motoric.  In some cases
intentions may play a role in coordinating sequences of very small scale purposive actions,
and in some cases motor representations may concern actions which are not very small scale.
The claim we wish to consider is only that, often enough, explaining the coordination of
sequences of very small scale actions appears to involve representations but not, or not
only, intentions.  To a first approximation, \emph{motor representation} is a label for
such representations.%
\footnote{%
Much more to be said about what motor representations are; for instance, see \citet{butterfill:2012_intention} for the view that motor representations can be distinguished by representational format.
}

\subsection{slide-14}
What do motor representations represent? An initially attractive, conservative
view would be that they represent bodily configurations and joint displacements,
or perhaps sequences of these, only.
However there is now a significant body of evidence that some motor representations
do not specify particular sequences of bodily configurations and joint displacements,
but rather represent outcomes such as the grasping of an egg or the pressing of a switch.
These are outcomes which might, on different occasions, involve very different bodily
configurations and joint displacements
(see \citealp{rizzolatti_functional_2010} for a selective review).

Such outcomes are abstract relative to bodily configurations and joint displacements
in that there are many different ways of achieving them.

Motor representations make very small scale actions the actions they are.
Which very small scale action you are performing or attempting to perform---for example,
which phoneme you are articulating or attempting to articulate---is a matter of which
outcomes are specified by the motor representations controlling your action.

\subsection{slide-15}
The Double Life of Motor Representations

Motor representations live a kind of double life. Although paradigmatically involved in performing
actions, they also occur in individuals who are not acting other than in observing others act.

That is, if you were to observe Ayesha reach for, grasp, transport and then place a pen, motor
representations would occur in you much like those that would also occur in you if it were you---not
Ayesha---who was doing this.

This is nicely illustrated for the case of speech by a study by Fadiga et al ...

\subsection{slide-16}
‘word listening produces a phoneme specific activation of speech motor centres’ \citep{Fadiga:2002kl}



‘Phonemes that require in production a strong activation of tongue muscles, automatically produce, when heard, an activation of the listener's motor centres controlling tongue muscles.’ \citep{Fadiga:2002kl}


\subsection{slide-17}
Good, but this stops short of showing that the motor activations
actually faciliatate speech recognition ...

How did they reach these conclusions?

\subsection{slide-19}
Inovlves tongue

\subsection{slide-20}
No tongue required

Given TMS to motor cortex tp amplify activity.
Prediction: MEP in tongue muscle stronger for ‘rr’ than ‘ff’.

\subsection{slide-22}
What are those motor representations doing here?

\subsection{slide-23}
‘Double TMS pulses were applied just prior to stimuli presentation to selectively prime the cortical activity specifically in the lip (LipM1) or tongue (TongueM1) area’
\citep[p.~381]{dausilio:2009_motor}

‘We hypothesized that focal stimulation would facilitate the perception of
the concordant phonemes ([d] and [t] with TMS to TongueM1), but that
there would be inhibition of perception of the discordant items
([b] and [p] in this case). Behavioral effects were measured via reaction
times (RTs) and error rates.’ \citep[p.~382]{dausilio:2009_motor}

\subsection{slide-24}
‘Effect of TMS on RTs show a double dissociation between stimulation
site (TongueM1 and LipM1) and discrimination performance between class
of stimuli (dental and labial). The y axis represents the amount of RT
change induced by the TMS stimulation. Bars depict SEM. Asterisks
indicate significance (p < 0.05) at the post-hoc (Newman-Keuls) comparison.’
\citep{dausilio:2009_motor}

\subsection{slide-26}
Motor representations concerning the goals of observed actions sometimes facilitate the identification of goals.

This leads to a further problem ...

\subsection{slide-27}
How is it that motor representations concerning the goals of observed actions sometimes
facilitate identification of goals?

\subsection{slide-28}
Just here puzzle arises.
We have seen that motor representations have
\textbf{content-respecting influences on thoughts}. It is the fact that one outcome rather
than another
is represented motorically which explains, at least in part, why the observer takes this
outcome (or a matching one) to be an outcome to which the observed action is directed. But how
could motor representations have content-respecting influences on thoughts? One familiar way
to explain content-respecting influences is to appeal to inferential relations. To illustrate,
it is no mystery that your beliefs have content-respecting influences on your intentions, for
the two are connected by processes of practical reasoning. But motor representation, unlike
belief and intention, does not feature in practical reasoning. Indeed, thought is
inferentially isolated from it. How then could any motor representations have
content-respecting influences on thoughts?

\subsection{slide-29}
It’s just here, I think, that it would be useful to appeal to experience.
But what kinds of experience?

\subsection{slide-30}
Thoughts about actions sometimes involve \emph{experiences revelatory of action}, that is,
experiences which provide the subject of experience with reasons for thoughts about the
goals of actions someone, herself or another, is performing.

We could understand how motor representations having content-respecting
influences on thought if
some experiences revelatory of action depended on motor representations of outcomes, and
did so in
such a way that which goals the experiences reveal was determined, at least in part, by which
outcomes are represented motorically.

But is there any evidence that this is true?

\subsection{slide-31}
Ideally we want to identify pairs of cases with these features: in each case there is an experience
revelatory of action; the two cases differ regarding what is represented motorically, but are
otherwise as similar as possible; and which goal is revealed in each case matches what is
represented motorically.

\subsection{slide-33}
Hemiplegic (HP)

\subsection{slide-34}
AHP patient

\subsection{slide-35}
‘Figure3
Examples of subjects’ right hand trajectory in Bimanual Circle-Line condition.
Note the increased ovalization for healthy controls (A) and for patients with AHP (B), but not for
hemiplegic (HP; C) or patients with motor neglect (MN; D).’

One source of such cases is research on anosognosia for hemiplegia. Patients with anosognosia for
hemiplegia will sometimes deny, and appear in some ways unaware of, a severe paralysis of one or
more limbs on one side of their bodies. Some such patients lack concurrent awareness of failures to
move their plegic limbs but do not suffer from severe sensory deficits or neglect, and cannot move
their hemiplegic limbs at all. For our purposes it is useful to focus only on these patients.

On the leading, best supported explanation, in these cases anosognosia for hemiplegia arises from deficiencies in monitoring action \citep{berti:2005_shared,berti:2008_motor}.
To illustrate, consider a patient who was asked to brush her hair holding a brush in her paralysed hand. Although she was unable to move the hand, she proceeded to move her head as if her hair was being brushed and then reported having successfully brushed her hair \citep[pp.\ 173--4]{berti:2008_motor}.
How could a deficit in monitoring action explain this?  When a subject with anosognosia for hemiplegia is asked to perform an action involving her hemiparetic limb, motor representations occur as they might do in ordinary subjects \citep{berti:2005_shared,garbarini:2012_moving}.
However, in ordinary subjects monitoring processes reliably ensure that any failures to act are detected; motor representations adjust accordingly \citep{Haggard:2005sc}.
By contrast, in these cases of anosognosia for hemiplegia, there is damage to the monitoring processes or to capacities underlying them.
A consequence is that motor representations are isolated from information relevant to failures to act.
This is why some patients with anosognosia for hemiplegia sometimes act as if their hemiparetic limbs were actually moving.

How is anosognosia for hemiplegia relevant to our concern with experiences revelatory of action?
Consider an anosognosic patient like those just mentioned and a patient with hemiplegia but no
anosognosia. Suppose each is asked to draw simultaneously with both hands, where the unaffected hand
was supposed to continuously draw a vertical line and the paralysed hand to continuously draw a
circle. There will be a difference in their experiences.
The anosognosic patient will sometimes judge that she is performing a bimanual action; this indicates that she has an experience which reveals the goal of drawing both lines and circles.
By contrast, the non-anosognosic patient with hemiplegia will report performing a unimanual action
and not the bimanual action, of course; this confirms that, as expected, she has no experience
revealing the goal of drawing both lines and circles. What could explain this difference in
experience between the two patients? The sensory information available to each patient should be the
same: after all, hemiplegic individuals can of course only actually move one hand, and the patients
we are concerned with do not have relevant sensory deficits. But there is a difference between the
patients' motor representations. In the anosognosic patient, deficient monitoring means that motor
representations should occur much like those that would occur were she not hemiplegic: there will be
motor representations concerning the movements of left and right hands. By contrast, in the
hemiplegic but non-anosognosic patient intact monitoring ensures that these motor representations do
not occur or do not persist: there will only be motor representations concerning the movements of
the unaffected hand. The predicted difference in motor representation can be confirmed by measuring
how straight the lines drawn are: in the anosognosic patient's case only, the attempted straight
line will show interference of the sort that, ordinarily, would be expected only if the other hand
were actually drawing a circle \citep{garbarini:2012_moving}. So comparing hemiplegic patients with
and without anosognosia yields a pair of cases fitting our criteria: there are differences in which
goals experiences reveal, and these differences appear to be determined by differences in what is
represented motorically.

\subsection{slide-36}
Motor representations concerning the goals of observed actions sometimes facilitate the identification of goals.

I take this to tell us something about the \textbf{function} of experiences associated with
very small scale actions.
\textbf{One function of these experiences is to enable us to identify the goals of actions, others’
and our own, in thought.}

\subsection{slide-37}
My question was,
What can we experience
when we encounter
very small-scale purposive actions?

My conclusion so far is this ...

\subsection{slide-38}
Experiences revelatory of action
shaped by motor representations.

Note: not making commitments on what these experiences are experiences of,
so not deciding between the Direct and Indirect Hypotheses.

Open question: how do the motor representations relate to the experiences.

One model: visual representation -> visual experience.

Another model: object indexes -> experience of objects.
(Motor representations structure experiences of events.)

\subsection{slide-40}
But my question was about the experiences of very small scale joint actions ...
Let me start by introducing you to the notion of a collective goal

\subsection{slide-41}
As I said at the start, my question is,
What do we experience of action when we act together?

I want to answer it by making a connection to the motoric.
But first let me introduce you to the notion of a collective goal ...

\subsection{slide-42}
Let me first explain something about this notion of a collective goal ...

Ayesha takes a glass and holds it up while Beatrice pours prosecco;
unfortunately the prosecco misses the glass and soak Zachs’s trousers.

\subsection{slide-43}
Here are two sentences, both true:

The tiny drops fell from the bottle.


The tiny drops soaked Zach’s trousers.


\subsection{slide-44}
The first sentence is naturally read *distributively*; that is, as specifying something
that each drop did individually.  Perhaps first drop one fell, then another fell.

\subsection{slide-45}
But the second sentence is naturally read *collectively*.
No one drop soaked Zach’s trousers; rather the soaking was something that the drops
accomplised together.

If the sentence is true on this reading, the tiny drops' soaking Zach’s trousers is not
a matter of each drop soaking Zach’s trousers.

\subsection{slide-46}
Now consider an example involving actions and their outcomes:

Their thoughtless actions soaked Zach’s trousers. [causal]

\subsection{slide-47}
This sentence can be read in two ways, distributively or collectively.
We can imagine that we are talking about a sequence of actions done
over a period of time, each of which soaked Zach’s trousers.
In this case the outcome, soaking Zach’s trousers, is an outcome of each action.

Alternatively we can imagine several actions which have this outcome collectively---as in
our illustration where Ayesha holds a glass while Beatrice pours.
In this case the outcome, soaking Zach’s trousers, is not necessarily an outcome of any of the
individual actions but it is an outcome of all of them taken together.
That is, it is a collective outcome.

(Here I'm ignoring complications associated with the possibility that some
of the actions collectively soaked Zach’s trousers while others did so distributively.)

Note that there is a genuine ambiguity here.
To see this, ask yourself how many times Zach’s trousers were soaked.
On the distributive reading they were soaked at least as many times as there are actions.
On the collective reading they were not necessarily soaked more than once.
(On the distributive reading there are several outcomes of the same type and each
action has a different token outcome of this type; on the collective reading there is a single token
outcome which is the outcome of two or more actions.)

Conclusion so far: two or more actions involving multiple agents can have outcomes
distributively or collectively.
This is not just a matter of words; there is a difference in the relation between
the actions and the outcome.

\subsection{slide-48}
Now consider one last sentence:

The goal of their actions was to fill Zach’s glass. [teleological]


\subsection{slide-49}
Whereas the previous sentence was causal, and so concened an actual outcome of some actions,
this sentence is teleological, and so concerns an outcome to which actions are directed.

\subsection{slide-50}
Like the previous sentence, this sentence has both distributive and collective readings.
On the distributive reading, each of their actions was directed to an outcome,
namely soaking Zach’s trousers.  So there were as many attempts on his trousers as there
are actions.
On the collective reading, by contrast, it is not necessary that any of the actions
considered individually was directed to this outcome;
rather the actions were collectively directed to this outcome.

Conclusion so far: two or more actions involving multiple agents can be collectively
directed to an outcome.

\subsection{slide-51}
Where two or more actions are collectively directed to an outcome, we will say that this
outcome is a *collective goal* of the actions.
Note two things.
First, this definition involves no assumptions about the intentions or other mental states
of the agents.  Relatedly, it is the actions rather than the agents which have a collective goal.
Second, a collective goal is just an actual or possible outcome of an action.

\subsection{slide-52}
Recall how Ayesha takes a glass and holds it up while Beatrice pours prosecco;
and unfortunately the prosecco misses the glass, soaking Zachs’s trousers.
Ayesha might say, truthfully, ‘The collective goal of our actions was not to soak Zach's trousers in
sparkling wine but only to fill this glass.’
What could make Ayesha’s statement true?

\subsection{slide-53}
As this illustrates,
some actions involving multiple agents are purposive in the sense that

\subsection{slide-54}
among all their actual and possible consequences,

\subsection{slide-55}
there are outcomes to which they are directed

\subsection{slide-56}
and the actions are collectively directed to this outcome

\subsection{slide-57}
so it is not just a matter of each individual action being directed to this outcome.

\subsection{slide-58}
In such cases we can say that the actions have a collective goal.

\subsection{slide-59}

\subsection{slide-60}
As what Ayesha and Beatrice are doing---filling a glass together---is a paradigm case of joint action, it might seem natural to answer the question by invoking a notion of shared (or `collective') intention.
Suppose Ayesha and Beatrice have a shared intention that they fill the glass.
Then, on many accounts of shared intention,

\subsection{slide-61}
the shared intention involves each of them intending that they, Ayesha and Beatrice, fill the glass;
or each of them being in some other state which picks out this outcome.

\subsection{slide-62}
The shared intention also provides for the coordination of their actions (so that, for example,
Beatrice doesn't start pouring until Ayesha is holding the glass under the bottle).  And
coordination of this type would normally facilitate occurrences of the type of outcome intended.
In this way, invoking a notion of shared intention provides one answer to our question about what
it is for some actions to be collectively directed to an outcome.

\subsection{slide-63}
Are there also ways of answering the question which involve psychological structures other than shared intention? In this paper we shall draw on recent discoveries about how multiple agents coordinate their actions to argue that the collective directedness of some actions to an outcome can be explained in terms of a particular interagential structure of motor representations.
Our actions having collective goals is not always only a matter of what we intend: sometimes it constitutively involves motor representation.

\subsection{slide-64}
Conjecture



Sometimes, when two or more actions involving multiple agents are, or need to be, coordinated:




Each represents a single outcome motorically, and


in each agent this representation triggers planning-like processes


concerning all the agents’ actions, with the result that


coordination of their actions is facilitated.



\subsection{slide-65}
What do we need?
(i) Evidence that a single outcome to which all the actions are directed is represented motorically.

\subsection{slide-66}
(ii) Evidence that this triggers planning-like processes,

\subsection{slide-67}
(iii) where these  concern all the agents' actions,

\subsection{slide-68}
and (iv) the existence of such representations facilitates coordination of the agents' actions.

\subsection{slide-70}
To test this conjecture, Corrado Sinigaglia and I teamed up with
Francesco della Gatta, Francesca Garbarini and Marco Rabuffetti.
We adapted a bimanual paradigm, the circle-line drawing paradigm, which has been
extensively employed for investigating bimanual interference (Franz et al, 1991).

When people have to simultaneously perform noncongruent movements,
such as drawing lines with one hand while drawing circles with the
other hand, each movement interferes with the other and line trajectories
tend to become ovalized. This “ovalization” has been described as a \textbf{bimanual coupling effect},
suggesting that motor representations for drawing circles can affect motor representations
for drawing lines (Garbarini et al. 2012; 2013a; 2015a; 2015b; Garbarini and Pia 2013;
Piedimonte et al. 2014).

In the key conditions of our adapted version of the circle-line drawing paradigm,
participants were asked to unimanually draw circles with their right hands
while observing either lines being unimanually drawn by a confederate (Garbarini et al, 2013b;
Garbarini et al, 2016).

We contrasted a Parallel Action task with a Joint Action task.
These tasks differed only in the instructions given.

In the Joint Action task participants were instructed to perform the task
together with the confederate, as if their two drawing hands gave shape to a single design.

In the Parallel Action task, participants were given no such instruction so that
they could draw in parallel, observing each other but not acting together.

If participants were to follow our instructions, their actions would have
the collective goal of drawing a circle and a line in the Joint Action task but not in
the Parallel Task.

Our conjecture entails that this collective goal could be represented motorically.
Accordingly, we predicted that there should be an interpersonal motor coupling effect.
This would result in greater ovalization of the lines drawn in the Joint Action task
than in the Parallel Action task ...

\subsection{slide-71}
And that was actually what we found.

Our hypothesis is that interpersonal motor coupling may occur when an individual is acting unimanually, providing she is acting jointly with another and not merely acting in parallel. This is because in joint action, but not in parallel action, an individual could represent motorically the collective goal of drawing both a circle and a line even if she is actually only drawing a line. Somewhat as in the case of individual bimanual action, so also in joint action: the motor representations of one hand’s drawing can influence the motor representations of the other hand’s drawing.  One difference in joint action, of course, is that the hands belong to different individuals.

\subsection{slide-74}
Conjecture



Sometimes, when two or more actions involving multiple agents are, or need to be, coordinated:


Of course the evidence I’ve provided is hardly sufficient for us to accept the
conjecture by itself.  It will only support the conjecture as part of a larger body
of evidence.

\subsection{slide-78}
As I said at the start, my question is,
What do we experience of action when we act together?

In conclusion, I first argued that
in performing and observing ordinary, individual very small scale actions such as
articulating a phonemic gesture, frowning or grasping a glass,
there are motor representations of the goals of these actions
which make the actions what they are
and shape the experiences associated with them in such a way as to
reveal the goals of these actions.

I then introduced the idea that in performing a very small scale joint action,
such as clinking glasses or playing a piano chord together,
the collective goal of our action is represented  motorically in each of us.

Putting the two ideas together, I suggest that in performing very small scale joint actions,
we have experiences of action much like those we would have if either of us were performing
the whole action alone.

That is, in performing very small scale actions together, we experience the unity of our action
and this experience is primitive in the sense that it does not depend on thinking or
practical reasoning about our action.

That is, there is an experience associated with performing very small scale joint actions
which does not involve experiencing
my part and your part separated: instead the experience involves action we are performing
as a whole.

This may have consequences for development.  A key developmental question is how humans
first come to understand joint action---that there are actions involving two people, in which
both contribute towards bring about some end.  If what I’ve been suggesting is right,
experience of the unity of our actions in performing very small scale joint actions could
play a role. (Of course we don’t yet know whether the experience is present early in development
or whether it emerges only with some reflection.)

It may also be relevant for accounts of shared agency which, like Michael Bratman’s,
aim to explain what shared agency is in terms of interconnected pratical reasoning.
[say why]

But the topic for today is empathy in and with groups. ...

\subsection{slide-79}
Motor representations shape experiences
associated with performing very small scale joint actions,
and these are a
simple form (or proto-form?) of empathy in groups.

I want to finsih by suggesting that the experiences
shaped by motor representations
associated with performing very small scale joint provide us with a simple case
of group empathy, one that enables us to know what we are doing when we clink
glasses together, pass a mug between us or play a chord together.

As I said at the start, a deeper form of empath seems to involve not these experiences
themselves but rather what happens when you interrupt them---when, for example, you
hold on to the mug too long in order to catch my eye, or move your glass just a fraction
away from mine, breaking the smooth coordination of our actions to  make mutually manifest
your discomfort.

But maybe the possibility of creating these moments of deep empathy depends, often enough, on
interrupting the
mundane and routine experiences involved in performing very small scale joint actions together.


    




%--- end paste
%---------------






\bibliography{$HOME/endnote/phd_biblio}



\end{document}
