 %!TEX TS-program = xelatex
%!TEX encoding = UTF-8 Unicode

\def \papersize {a6paper}
%\def \papersize {a4paper}
%\def \papersize {letterpaper}

\documentclass[14pt,\papersize]{extarticle}
%\documentclass[12pt,\papersize]{extarticle}
% extarticle is like article but can handle 8pt, 9pt, 10pt, 11pt, 12pt, 14pt, 17pt, and 20pt text

\def \ititle {Origins of Mind: Lecture Notes}
\def \isubtitle {Lecture 01}
%comment some of the following out depending on whether anonymous
\def \iauthor {Stephen A.\ Butterfill}
\def \iemail{s.butterfill@warwick.ac.uk% \& corrado.sinigaglia@unimi.it
}
%\def \iauthor {}
%\def \iemail{}
%\date{}

\input{$HOME/Documents/submissions/preamble_steve_paper3}
%\input{$HOME/Documents/submissions/preamble_steve_report2}

%no indent, space between paragraphs
\usepackage{parskip}

%comment these out if not anonymous:
%\author{}
%\date{}

%for e reader version: small margins
% (remove all for paper!)
\geometry{headsep=2em} %keep running header away from text
\geometry{footskip=1.5cm} %keep page numbers away from text
\geometry{top=1cm} %increase to 3.5 if use header
\geometry{bottom=2cm} %increase to 3.5 if use header
\geometry{left=1cm} %increase to 3.5 if use header
\geometry{right=1cm} %increase to 3.5 if use header

% disables chapter, section and subsection numbering
\setcounter{secnumdepth}{-1} 

%avoid overhang
\tolerance=5000

%\setromanfont[Mapping=tex-text]{Sabon LT Std} 


%for putting citations into main text (for reading):
% use bibentry command
% nb this doesn’t work with mynewapa style; use apalike for \bibliographystyle
% nb2: use \nobibliography to introduce the readings 
\usepackage{bibentry}

%screws up word count for some reason:
%\bibliographystyle{$HOME/Documents/submissions/mynewapa} 
\bibliographystyle{apalike} 


\begin{document}



\setlength\footnotesep{1em}






%--------------- 
%--- start paste


      
\title {Commentary on Wolfgang Prinz and Michael Graziano}
 
 
 
\maketitle
 
\subsection{title-slide}
I have a series of questions which I want to put to Wolfgang Prinz and
Michael Graziano.
 
The broad thrust of my questions might be put like
this.  
Both theories involve taking as primary a third-person perspective of 
subjectivity or awareness, a perspective of it as seen by an onlooker with a
side on view.  
More than this, they aim to show that other perspectives, and in particualar
a first person perspective, are dispensable in the sense that we can
construct these from a third-person perspective.
Thus Wolfgang talks about importing from a third-person perspective on
another’s experiences and actions, and Michael explains awareness in terms
of a model of attention which describes attention as a relation
between a subject and a thing.
My aim in what follows is to explore reasons for thinking that 
other perspectives,
the first-person perspective and the perspective we adopt
when we do things together,
are indispensable.
It is not that I think another perspective has primacy.
My guess is that none are primary and that all are indispensable in the 
sense that you cannot construct any one from the others.
 
OK, now for some questions.
 
--------
\subsection{slide-3}
My first question is, What about us?
Let me explain
 
\subsection{slide-4}
‘minds [are] open systems whose basic makeup is determined through social interaction and communication’


            
Prinz (2012, p. 33)

 
\subsection{slide-5}
Wolfgang Prinz offers a contrast between two directions.
It seems to me that there is an additional possibility.
 
\subsection{slide-6}
Couldn't it be that before either of us can think about 
the actions and mental states of individuals, we are acting
together?
And not only acting together but making use of the machinery
for explanation and prediction which Wolfgang Prinz says 
is needed for the construction of a will?
 
\subsection{slide-7}
I'm not sure but I think
I think Wolfgang suggests that we need to have individual agents
in place before we can act together.
I think this because he describes
 
‘the will as an individual device to serve ...’ ‘the requirements of collective control’


          
Prinz (2012, p. 137)

 
But the notion of common coding seems to invite us to think in another way ...
 
\subsection{slide-8}
To say that there is common coding is to say that there is
 
‘a shared space of representation that uses the same set of 
representational dimensions for coding one’s own action and 
foreign perception’
 
Common coding doesn't just mean that I can represent your actions 
and my actions in the same way.
It also means that I can represent things we do together in the ordinary
way that I can represent actions.
Suppose we move a table together.
Given common coding, I can represent our actions in the same
way as I would represent your actions or my actions if either of us
were doing this alone.  Actions are represented in terms of events.
 
So, as I understand it,
common coding suggests it is coherent to suppose that 
Animals could act together before they can recognise any individual 
agents at all.
 
Indeed, complementary mirroring seems to be a case of animals acting
together without necessarily recognising any individual agents.
 
In that case, why is it right to think of the roots of subjectivity
as involving a transition from you to me rather than from us to you and me?
 
\subsection{slide-9}
My second question is about awareness, Michael Graziano’s attention
schema and development.
 
\subsection{slide-10}
Let me start with some background on the question for Graziano.
His view is that there is a model of attention which allows me 
to make decisions about whether I, or anyone else, is aware of something:
 
(1) ‘my brain [...] constructs a set of information, A, that allows me to 
conclude that I am aware of the content [X].’
            


            
Graziano (2013, p. 70)

 
It's important for what follows that this model is used both in
making decisions about oneself and about what others are aware of.
 
Of course, he also goes further and says that ...
 
\subsection{slide-11}
(2) ‘awareness is an attention schema.’
              


              
Graziano (2013, p. 69)

 
The idea is that being aware of something is a matter of the
model of attention attaching a feature to it.
That is ...
 
\subsection{slide-12}
I.e. ‘the property of awareness is ... a chunk of information, that can 
be bound to the larger 
object file.’
              


              
Graziano (2013, p. 21)

 
My first question is just about step 1.  Forget about whether this is
awareness---I'll come to that in a moment.
 
\subsection{slide-13}
A range of findings suggest that two-year-olds make judgements about
awareness which, to adults, seem wrong.
 
\subsection{slide-14}
In fact children's judgement about awareness seem to change over a year or
longer, gradually becoming more adult like
 
Taken at face value, this is evidence that young children have a model of
attention that is quite different from typical adults’.
 
Of course, you might suppose that children’s performance on these tasks 
does not truly reflect their underlying model of attention but is a
consequence of some extraneous factor.
But given the variety of methods used, it would be hard to take this line.
 
So I think we have strong reasons to hold that, at some age, children’s
judgements about what others are aware of are based on a model of attention
which is strikingly different from most adults’ model.
 
\subsection{slide-15}
This has an interesting consequence for Michael Graziano’s theory. ...
 
(1) ‘my brain [...] constructs a set of information, A, that allows me to 
conclude that I am aware of the content [X].’
            


            
Graziano (2013, p. 70)

 
(2) ‘awareness is an attention schema.’
            


            
Graziano (2013, p. 69)

 
The idea is that being aware of something is a matter of the
model of attention attaching a feature to it.
That is ...
 
I.e. ‘the property of awareness is ... a chunk of information, that can 
be bound to the larger 
object file.’
            


            
Graziano (2013, p. 21)  

 
2 year olds have a different attention schema from that of most adults’.
So they attach the property of awareness to different object files
than adults do.
In particular, if you blindfold a 2-year-old at the start of a hiding game,
then take the blindfold off before actually hiding the object,
the 2-year-old will not attach the property of awareness to the object 
that is hidden and so will not be aware of this object.
 
Let me summarise this reasoning ...
 
\subsection{slide-16}
one- and two-year-olds’ representations of awareness differ systematically from most adults’

                
one- and two-year-olds’ attention schema differs systematically from most adults’

                
Conditions under which 1- and 2-year-olds’ are aware of a thing differ systematically from conditions under which most adults are aware of a thing.
 
There isn't an objection in view yet.
After all, lots of scientific theories have surprising consequences.
 
But I do think there is an objection.
This is because awareness typically affects behaviour, as deficits like
neglect show.
If (3) were true, one- and two-year-olds should fail to search for, or talk
about objects in cases where they are unaware of them.  In fact, they should
show signs of being quite significantly impaired.  But as far as I know
they don’t.
 
\subsection{slide-17}
You can prevent a two-year-old from finding the chocolate by hiding it while
she isn't looking but not by blindfolding her before you hide it.
 
\subsection{slide-18}
My third question is about phenomenology.
 
\subsection{slide-19}
NB So there is an assumption that we can explain subjectivity  without
thinking about the qualitative aspects of expeirence.
 
I want to suggest that there are reasons to doubt that we can separate
subjectivity from the qualitative aspects of experience.
I think this will make trouble for a central claim that both Prinz and 
Graziano make in form or another.
 
\subsection{slide-20}
Consider this question:
 
What is like you?
 
Clearly it can't be a subject of experience or an agent since the point
of the ‘like you’ perspective is to explain how these things are 
constructed.
Instead it must be a body.
This is clear when we consider the constructivist claims both Prinz
and Graziano make.
 
\subsection{slide-21}
In both cases, Prinz and Graziano, the theories are constructive in the
sense that they do not regard the theories as answerable to anything
that is awareness or subjective.
 
\subsection{slide-22}
I want to remind you about this observation from Brentano ...
 
At the same time, hearing a tone does not involve hearing a subject.
Likewise, seeing a tree does not involve seeing a perceiver.
So how is the subject ‘entailed in the act’?
 
The answer surely involves fact that hearing and seeing typically
result in experiences which are perspectival.
 
When you see a tree or hear a tone, you do so from a point of view.
The point of view is not represented, it is not part of what you perceive.
But the expeirence is organised around this point of view.
 
\subsection{slide-23}
We can bring this out by contrasting two types of change:
you rotate or an object rotates.
Although the two rotations may result in your seeing the same aspect of the
object, they involve a different alteration in your expeirence.
 
\subsection{slide-24}
So it seems that some experiences implicitly specify their subjects
in virtue of being organised around a point of view.
I think this leads to a dilemma.
 
\subsection{slide-25}
Phenomenology is like the bad fairy in cinderella.
No one wants to invite her to the party, 
but not inviting will only make things worse for you or your daughters.
 
\subsection{slide-26}
My fourth question is about the idea that expeirence matters for objectivity,
the fact that our thoughts and theories can be false.
 
\subsection{slide-28}
How could visual expeirence (say) of an apple enable you to think
demonstrative thoughts about it?
 
\subsection{slide-29}
On Graziano’s view, the answer is that, strictly speaking,
visual expeirence does not 
enable you think demonstrative thoughts about objects.
 
And Graziano has to say this because on his theory there is no way that 
awareness could enable you to think demonstrative thoughts about an
object.  After all, on his view awareness is just another layer of 
information processing.
I'm less sure how things go on Prinz’ theory, but as far as I can tell
this view also can't explain how visual expeirence 
enables you think demonstrative thoughts about objects.
 
\subsection{slide-30}
I want to focus on the intution, not the fact.
It is not a theoretical claim but a feature of commonsense thinking
about visual experience that visual experience enables us to 
think demonstrative thoughts about objects.
Graziano and Prinz can deny that this is a fact, but their theories 
need to explain the intuition.
So let me ask another question,
How can either theory explain the intuition?
 
\subsection{slide-31}
I think both explain it in roughly the same way.
On Michael's view, awareness is modelled as a relation between a subject, S,
and a thing, X.
So it's clear that I would think of awareness as putting me in touch
with X, and so being able to think about it.
Likewise, on Prinz’ view, I come to think of myself as having mental states
by first ascribing them to others, and in ascribing them to others I am
relating others to objects.
So again the ‘intuition of mental access’ can be explained even if 
the intuition is, strictly speaking, wrong.
 
But I think these theories explain too much ...
 
\subsection{slide-33}
The gist of these questions is this.
Let’s stop taking one perspective or aspect of subjective experience 
to be primary and trying to construct others. 
I suspect that the first-person perspective, phenomenal aspects of 
expeirence and the social are all indispensable.
What I have learned is not that something is primary, but rather
that things typically thought about as narrowly first-person phenomena are
actually intrinscially bound up with the social.
It's not about deciding what is primary and what is constructed;
it's about tracing the ways in which different perspectives are interdependent.
 
\subsection{slide-34}
Graziano’s model is additive in a strong sense.
There are these three components, and you can have the 
outer two, S or X, independently of awareness.
(Awareness is something which involves a model of 
the relation between S and X.)
This is very clear from the book:
 
‘Suppose that you are looking at a green object and have a conscious 
experience of greenness. ...
 
\subsection{slide-35}
So awareness is a chunk of information that can be added (or not)
to a state, and when you add that information you get awareness.
 
Graziano’s model is additive in a strong sense.
 
One tiny question I had was, How can this view accomodate situations where
you are aware of only some of the features of an object although
you are receiving perceptual information about many of its features?
Here is an example ...
 
CUT THIS QUESTION.  The model must be able to deal with such cases because
after all you are attending to different things!
 
\subsection{slide-36}
I couldn't find an apple so I'll have to make do with a green square
Suppose you attend to the shape and let’s say you’ve seen lots of these
so you’re really focussed on the shape.
You might say, sincerely, ‘I was aware of the shape but not the colour.’
Even so, you are computing its categorical colour properties, as is 
shown by the fact that you can get odd-ball effects for changes in 
categorical colour properties.
So X includes inforamation about shape plus information about categorical 
colour.
But on the additive model, A gets added to X or not.
Given the additive nature of the model,
how can we distinguish this case, in which there is awareness shape 
from a case in which you are aware of categorical colour and shape?
Is the idea, for example, that the awareness feature gets bound to 
something other than an object, perhaps a partial object?
 
So this was my question about whether we have a full explanation of how
people make decisions about awareness.
 
I also want to note a question about how the model of attention 
changes across development.
 
\subsection{slide-37}
Recall the two steps ...
 
(1) ‘my brain [...] constructs a set of information, A, that allows me to 
conclude that I am aware of the content [X].’
            


            
Graziano (2013, p. 70)

 
(2) ‘awareness is an attention schema.’
            


            
Graziano (2013, p. 69)


            

            
 
 
It is striking that children’s understanding of awareness appears to 
develop quite gradually over the first years of life.
For example, some findings indicate that two-year-olds treat
people as unanware of things if they have been bindfolded at some point
during a hiding episode, even if their eyes were open at the crucial 
moment (Dunham et al (2000)).
 
Reluctant to infer from judgements direct to the model, but plausible 
that there are changes.
 
Consequence of Graziano's theory is that these changes should be reflected
in the nature of children’s awareness.
 
\subsection{slide-38}
S + A + X
 
Experience involves a subject, S, being aware, A, of a thing, X.
Roughly speaking, 
Prinz focusses on the subject of experience, S,
and Graziano about the awareness, A, the thing in the middle.
(Graziano says his theory is about the thing that ‘lies between the “I” and 
the “X,”’ (p. 30)).
I want to start with two questions about the X, the first for Graziano
and the second for Prinz.
 

    

%--- end paste
%--------------- 
 





\bibliography{$HOME/endnote/phd_biblio}



\end{document}