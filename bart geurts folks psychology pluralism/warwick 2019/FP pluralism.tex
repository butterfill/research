\documentclass[12pt]{article}

\newcommand{\E}[1]{\emph{#1}}
\newcommand{\Q}[1]{``#1''}

\title{Some notes on folk-psychological pluralism}
\author{}
\date{}

\begin{document}

\maketitle

\begin{enumerate}

\item \E{General.} I take it that our general theme is folk psychology (FP), which is a practice that enables our everyday social interactions. Mental-state attribution is part of that practice, and it may be implicit or explicit. I think we should focus on belief attribution, if only because most of the literature is about that, but other mental states shouldn't be ignored altogether, if only because belief attribution is likely to be special in various ways.
\par I suppose that social interaction usually involves being engaged in coordinated activities, and not just detached observation (as in the standard false-belief paradigm), and that, typically, these activities include communication.

\item \E{Aspects of belief.} If belief attribution is our central topic, one way of approaching it would be to look at various theories of \E{belief} and consider their relevance for our purposes. My initial impression is that the following isms are the most useful: dispositionalism, functionalism, materialism. Between these three, I would say that dispositionalism is the most important, followed by the second: the mental states we attribute to each other are associated with characteristic patterns of behaviour, which interact with each other in sundry ways. The fact that mental states are taken to be situated in this or that part of the body seems of relatively little interest, because it doesn't add much to the practical explanatory power of FP.  

\item \E{Normativity.} I'm sure that many of the beliefs that I would be disposed to attribute to you have a normative aspect, of some sort or other, two of which are the following. First, there are lots of beliefs I would attribute to you because you are a neurotypical human, male, adult, Englishman, philosopher, and so on. Due to these circumstances, I take it that you \E{should} believe what you see, that you know what \Q{Brexit} means, that you know that Plato was Greek, and so on. Secondly, I take it that, by and large, your beliefs respect some logic or other. For example, if you believe that Fred is taller than Barney, you must believe that Barney isn't taller than Fred. 
\par Putting together this point with the previous one, it seems to follow that, to a significant extent, the beliefs we attribute to others are behavioural dispositions that they are supposed to have.
\par The beliefs of the belief attributor are important here, especially if the norms in question are of a moral kind. E.g., if someone is vehemently opposed to incest, she might be more likely to attribute to others the belief that incest is wrong than if she doesn't have strong feelings on the matter.

\item \E{Language and communication} are potentially relevant in a number of ways, including the following:
\begin{enumerate}\itemsep=0ex
\item We use linguistic means to explicitly attribute mental states to others and ourselves: mental-state verbs, evidentials, intonation, etc.
\item Even if it is controversial what the primary function of communication is, it is uncontroversial that our utterances typically convey intentions, beliefs, and so on, and speakers' utterances \E{commit} them to having these states. E.g., an utterance of \Q{Tomorrow is Tuesday} will normally commit the speaker to having the belief that tomorrow is Tuesday.
\item Mental-state attribution may be contingent on the availability of linguistic means (words, grammatical forms, \ldots) for attributing the states in question.
\par Which raises the question whether, according to FP, creatures without language can have beliefs.
\item \E{Common ground} is one of the key concepts in pragmatics, and in theories of social interaction more generally (though usually not under this name). Establishing common ground (\Q{grounding}) is closely related to the attribution of beliefs and intentions, and on some accounts it \E{is} is a form of mental-state attribution (and presumably the most important one).
\end{enumerate}

\item \E{Introspection.} I know what it is like to be nervous, apprehensive, irritated, etc., and if I attribute any of these states to you, I assume that it is the same for you. But what is it like to believe that tomorrow is Tuesday? I don't know, and therefore I wouldn't know what it is like for you. Intentions are the same as beliefs in this respect. If I'm not the only one to have these intuitions, they might reflect a distinctive feature of the FP of belief and intention.

\item \E{Memory.} Beliefs are primarily standing states, which persist by default. However, we also tend to forget about our beliefs, some of them are easier forgotten than others, and this is reflected in the FP of memory. 

\item \E{Compartmentalisation.}  Whenever Fred is with Wilma, he is convinced that she is the perfect woman for him, but when he is with Betty, he feels the same about her. (And when he is on his own, he doesn't know.) Fred's beliefs seem to be compartmentalised: different beliefs for different occasions. A simpler case might be of a person who takes both \E{p} and \E{q} to be true even if \E{p} entails not-\E{q} (in which case the person might be said to be in two conflicting belief states at the same time). Belief compartmentalisation is an important ingredient in Stalnaker's account of deductive reasoning.

\end{enumerate}
%\renewcommand{\refname}{\normalsize References}
%\small
%\bibliography{biblio}

\end{document}