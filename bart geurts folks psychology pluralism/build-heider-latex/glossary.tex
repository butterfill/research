\longnewglossaryentry{A task}{
        name={A task}
      }{Any false belief task that typically developing children tend to fail until
around three to five years of age.
}

\longnewglossaryentry{Affect Heuristic}{
        name={Affect Heuristic}
      }{In the context of moral psychology, the Affect Heuristic is this principle:
‘if thinking about an act [...] makes you feel bad [...], then judge that it is 
morally wrong’ \citep{sinnott:2010_moral}.  These authors hypothesise that the 
Affect Heuristic explains moral intuitions. 

A different (but related) Affect Heurstic has also be postulated to explain how
people make judgements about risky things are: The more dread you feel when imagining an 
event, the more risky you should judge it is (see \citealp{pachur:2012_how},
which is discussed in \ref{unit:affect_heuristic}).
}

\longnewglossaryentry{altercentric interference}{
        name={altercentric interference}
      }{The unintentional or nonpurposive influence of another’s beliefs (or other
mental states) on your own beliefs (or corresponding mental states).
}

\longnewglossaryentry{Assumption of Representational Connections}{
        name={Assumption of Representational Connections}
      }{The assumption that the transition from early-developing forms of representation to belief or knowledge
involves operations on the early-developing forms of representation which transform their contents into (components of) the contents of knowledge states.
The Assumption of Representational Connections is implicitly required by many theories about the developmental emergence of knowledge, but not by the view that development is rediscovery.
}

\longnewglossaryentry{automatic}{
        name={automatic}
      }{On this course, a process is \emph{automatic} just if whether or not it
occurs is to a significant extent independent of your current task,
motivations and intentions. To say that \emph{mindreading is automatic} is
to say that it involves only automatic processes.  The term `automatic' has
been used in a variety of ways by other authors: see
\citet[p.~22]{moors:2014_examining} for a one-page overview,
\citet{moors:2006_automaticity} for a detailed theoretical review, or
\citet{bargh:1992_ecology} for a classic and very readable introduction
}

\longnewglossaryentry{Autonomy Thesis}{
        name={Autonomy Thesis}
      }{The thesis that ‘ethical theory is autonomous of other forms of intellectual inquiry in that 
its fundamental deliberations do not depend on input from other subjects, especially empirical ones’ \citep[p.~259]{rini:2013_making}. See \citep[AK 4:425--6]{kant:1870_grundlegung} for an endorsement. 
}

\longnewglossaryentry{binding foundations}{
        name={binding foundations}
      }{Categories of moral concern linked to social needs; these are often taken to be 
betrayal/loyalty, subversion/authority, and impurity/purity \citep{graham:2011_mapping}.
}

\longnewglossaryentry{canonical model of minds and actions}{
        name={canonical model of minds and actions}
      }{A model specified by a \gls{canonical theory of the mental}.
}

\longnewglossaryentry{Causal Theory of Action}{
        name={Causal Theory of Action}
      }{According to this view, an event is action ‘just in case it has a certain sort of 
psychological cause’ \citep[p.~361]{bach:1978_representational}.
}

\longnewglossaryentry{characteristically consequentialist}{
        name={characteristically consequentialist}
      }{According to Greene, a judgement is \emph{characteristically consequentialist}
(or *characteristically utilitarian*) if it is one in
‘favor of characteristically consequentialist conclusions (eg, “Better to save more lives”)’ \citep[p.~39]{greene:2007_secret}.
According to \citet[p.~365]{gawronski:2017_consequences}, ‘a given judgment cannot be categorized
as [consequentialist] without confirming its property of being sensitive to consequences.’
}

\longnewglossaryentry{characteristically deontological}{
        name={characteristically deontological}
      }{According to Greene, a judgement is \emph{characteristically deontological} 
if it is one in
‘favor of characteristically deontological conclusions (eg, “It’s wrong despite the benefits”)’ \citep[p.~39]{greene:2007_secret}.
According to \citet[p.~365]{gawronski:2017_consequences}, ‘a given judgment cannot be categorized
as deontological without confirming its property of being sensitive to
moral norms.’
}

\longnewglossaryentry{cognitively efficient}{
        name={cognitively efficient}
      }{A process is \emph{cognitively efficient} to the degree that it does not consume working
memory and other scarce cognitive resources.
}

\longnewglossaryentry{communicative intention}{
        name={communicative intention}
      }{A communicative intention is an intention to communicate, as opposed to an intention to achieve some extra-communicative end (such as getting you to pass me the salt).
}

\longnewglossaryentry{computational description}{
        name={computational description}
      }{A computational description of a system or ability specifies what the thing is for and how it achieves this.
\citet{Marr:1982kx} distinguishes the computational description of a system from \gls{representations and algorithms} and its hardware implementation.
}

\longnewglossaryentry{Conjecture MP}{
        name={Conjecture MP}
      }{In the first nine months of life,
all proper pure goal tracking is explained by the Motor Theory.
Other pure goal-tracking processes emerge later in development.
Further, a mere target-tracking process is also present in these infants.
This process is identical to \gls{perceptual animacy} in adults.
And appearances that these infants’ pure goal-tracking abilities are not limited by
what they can represent motorically are misleading:
they are due to mistaking mere target tracking for proper goal tracking.
}

\longnewglossaryentry{construct}{
        name={construct}
      }{A factor postulated by a theory with the aim of explaining patterns of behaviour.
Examples of constructs include moral conviction, moral disengagement and the moral foundations
from Moral Foundations Theory.
}

\longnewglossaryentry{core knowledge}{
        name={core knowledge}
      }{For an individual to have core knowledge concerning a domain such as physical objects, actions or minds is for her to have a \gls{core system} specifically for this domain.
For someone to have core knowledge of a particular principle or fact is for her to have a \gls{core system} where
either the core system includes a representation of that principle or else the principle plays a special role in describing the core system.
Core knowledge is not knowledge, and you can have core knowledge of things that are untrue
(for this reason \citet[p.~10]{carey:2009_origin} recommends the term ‘core cognition’ for states of core knowledge).
}

\longnewglossaryentry{core system}{
        name={core system}
      }{This course uses a nonstandard, minimally informative notion of core system
on which a ‘core system’ for a particular domain is simply  whatever it is
that underpins the earliest abilities infants manifest in that domain (see
\cref{sec:core-knowledge-minimal-view}).
This allows that core systems may lack uniformity across domains and unity
within a domain: that is, different kinds of system may qualify as ‘core’ in
different domains, and a core system may comprise two or more largely
distinct systems (see \cref{sec:paradox-lost}).

However, core systems are standardly identified by giving a list of features.
The lists vary between researchers and times.
\citet[p.~520]{Carey:1996hl} assert that core systems are largely \gls{innate}, informationally encapsulated (that is, their operations are largely unaffected by things you know or believe, and by core knowledge in other core systems), largely unchanging over the course of development (so adults and infants alike have the same core systems).
They also say that the inputs to core systems are the outputs of perceptual systems, so that architecturally core systems in human adults occupy a position between perception and knowledge.
Finally, core systems are also held to arise from systems already present in the evolutionary ancestors of modern humans.
\citet{carey:2009_origin} adds that 
the representations in core systems are \glspl{iconic representation}.
}

\longnewglossaryentry{David}{
        name={David}
      }{‘David is a great transplant surgeon. Five of his patients need new parts—one
needs a heart, the others need, respectively, liver, stomach, spleen, and
spinal cord—but all are of the same, relatively rare, blood-type. By chance,
David learns of a healthy specimen with that very blood-type. David can take
the healthy specimen's parts, killing him, and install them in his patients,
saving them. Or he can refrain from taking the healthy specimen's parts,
letting his patients die’
\citep[p.~206]{thomson:1976_killing}.
}

\longnewglossaryentry{debunking argument}{
        name={debunking argument}
      }{A \emph{debunking argument} aims to use 
facts about why people make a certain judgement 
together with
facts about which factors are morally relevant 
in order to undermine the case for
accepting it. \citet[p.~2607]{konigs:2019_experimental} provides a 
useful outline of the logic of these arguments (which he calls 
‘arguments from moral irrelevance’):
‘when we have different intuitions about similar moral cases, we take this
to indicate that there is a moral difference between these cases. This is
because we take our intuitions to have responded to a morally relevant
difference. But if it turns out that our case-specific intuitions are
responding to a factor that lacks moral significance, we no longer have reason
to trust our case-specific intuitions suggesting that there really is a moral
difference. This is the basic logic behind arguments from moral irrelevance’
\citep[p.~2607]{konigs:2019_experimental}.
}

\longnewglossaryentry{Developmental Motor Conjecture}{
        name={Developmental Motor Conjecture}
      }{In the first nine months of life, all pure goal tracking is explained by the 
\gls{Motor Theory of Goal Tracking}. Other goal-tracking processes emerge later in development.
}

\longnewglossaryentry{directed}{
        name={directed}
      }{For an action to be \emph{directed} to an outcome is for the action to happen in order to bring that outcome about.}

\longnewglossaryentry{distal outcome}{
        name={distal outcome}
      }{The \glspl{outcome} of an action can be partially ordered by the cause-effect
relation. For one outcome to be more \emph{distal} than another is for it to
be lower with respect to that partial ordering. To illustrate, if you kick
a ball through a window, the window’s breaking is a more distal outcome than
the kicking.
}

\longnewglossaryentry{doctrine of double effect}{
        name={doctrine of double effect}
      }{‘the thesis that it is sometimes permissible to bring about by oblique intention what one may not directly intend’ \citep[p.~7]{foot:1967_problem}.}

\longnewglossaryentry{domain specific}{
        name={domain specific}
      }{A process is domain specific to the extent that there are limits on the range of functions
its outputs typically serve.  Domain-specific processes are commonly contrasted with general-purpose
processes. 
}

\longnewglossaryentry{Drop}{
        name={Drop}
      }{A dilemma; also known as \emph{Footbridge}. A runaway trolley is about to run over and kill five people. 
You can hit a switch that will release the bottom of a footbridge and
one person will fall onto the track. The trolley will hit this person, 
slow down, and not hit the five people further down the track. 
Is it okay to hit the switch? 
}

\longnewglossaryentry{dual process theory of mindreading}{
        name={dual process theory of mindreading}
      }{A theory on which mindreading involves two or more processes which are
distinct in this sense: the conditions which influence whether one
mindreading process occurs differ from the conditions which influence
whether another occurs.
(For background on dual process theories in social cognition generally,
\citet{sherman:2014_dual} is an excellent collection of essays.)
}

\longnewglossaryentry{dual-process theory}{
        name={dual-process theory}
      }{Any theory concerning abilities in a particular domain on which those
abilities involve two or more processes which are distinct in this sense:
the conditions which influence whether one mindreading process occurs differ
from the conditions which influence whether another occurs.
}

\longnewglossaryentry{Edward}{
        name={Edward}
      }{‘Edward is the driver of a trolley, whose brakes have just failed. On the
track ahead of him are five people; the banks are so steep that they will not
be able to get off the track in time. The track has a spur leading off to
the right, and Edward can turn the trolley onto it. Unfortunately there is
one person on the right-hand track. Edward can turn the trolley, killing the
one; or he can refrain from turning the trolley, killing the five’
\citep[p.~206]{thomson:1976_killing}.
}

\longnewglossaryentry{extra communicative purpose}{
        name={extra communicative purpose}
      }{Any purpose that might in principle be achieved without communication at all
}

\longnewglossaryentry{extra-communicative intention}{
        name={extra-communicative intention}
      }{Any intention that might in principle be achieved without communication at all.
}

\longnewglossaryentry{Far Alone}{
        name={Far Alone}
      }{‘I alone know that in a distant part of a
foreign country that I
am visiting, many children are drowning, and I alone can save one of them.
To save the one, all I must do is put the 500 dollars I carry in my pocket into a
machine that then triggers (via electric current) rescue machinery that will
certainly scoop him out’
\citep[p.~348]{kamm:2008_intricate}
}

\longnewglossaryentry{fast}{
        name={fast}
      }{A \emph{fast} process is one that is to 
to some interesting degree \gls{cognitively efficient}
(and therefore likely also some interesting degree \gls{automatic}).
These processes are also sometimes
characterised as able to yield rapid responses.

Since automaticity and cognitive efficiency are matters of degree, it is only strictly
correct to identify some processes as faster than others. 

The fast-slow distinction has been variously characterised in ways that do not 
entirely overlap (even individual author have offered differing characterisations
at different times; e.g. 
\citealp{kahneman:2013_thinking}; \citealp{morewedge:2010_associativea}; 
\citealp{kahneman:2009_conditions}; \citealp{kahneman:2002_maps}): 
as its advocates stress, 
it is a rough-and-ready tool rather than an element in a rigorous theory.
}

\longnewglossaryentry{Frank}{
        name={Frank}
      }{‘Frank is a passenger on a trolley whose driver has just shouted that the
trolley's brakes have failed, and who then died of the shock. On the track
ahead are five people; the banks are so steep that they will not be able to get
off the track in time. The track has a spur leading off to the right, and
Frank can turn the trolley onto it. Unfortunately there is one person on the
right-hand track. Frank can turn the trolley, killing the one; or he can
refrain from turning the trolley, letting the five die’
\citep[p.~207]{thomson:1976_killing}.
}

\longnewglossaryentry{goal}{
        name={goal}
      }{A \emph{goal} of an action is an outcome to which it is directed.}

\longnewglossaryentry{goal-state}{
        name={goal-state}
      }{an intention or other state of an agent which links an action of hers to a 
particular goal to which it is directed.
}

\longnewglossaryentry{Gricean model}{
        name={Gricean model}
      }{On the Gricean model of communication,
to produce a communicative action involves acting with an intention to provide someone with evidence of an intention with the further intention of thereby fulfilling that intention.
And to comprehend a communicative action is to know that the communicator has such intentions.
}

\longnewglossaryentry{habitual process}{
        name={habitual process}
      }{A process underpinning some instrumental actions which obeys
*Thorndyke’s Law of Effect*: 
‘The presenta­tion of an effective [=rewarding] outcome following an action [...] rein­forces 
a connection between the stimuli present when the action is per­formed and the action itself 
so that subsequent presentations of these stimuli elicit the [...] action as a response’ 
\citep[p.48]{Dickinson:1994sm}.
}

\longnewglossaryentry{habituation}{
        name={habituation}
      }{Habituation is used to test hypotheses about which events are interestingly different to an infant.
In a habituation experiment, infants are shown an event repeatedly until it no longer holds their interest, as measured by how long they look at it.
The infants are then divided into two (or more) groups and each group is shown a new event.
How much longer do they look at the new event than at the most recent presentation of the old event?
This difference in looking times indicates \emph{dishabituation}, or the reawakening of interest.
Given the assumption that greater dishabituation indicates that the old and new events are more interestingly different to the infant, evidence from patterns of dishabituation can sometimes support conclusions about patterns in how similar and different events are to infants.
}

\longnewglossaryentry{heuristic}{
        name={heuristic}
      }{A \emph{heuristic} links an inaccessible attribute to an accessible attribute such that, within a limited but useful range of situations, someone could track the inaccessible attribute by computing the accessible attribute.}

\longnewglossaryentry{iconic representation}{
        name={iconic representation}
      }{A representation is {iconic} if parts of the representation represent parts of the thing represented.
Pictures are paradigm examples of iconic representations.
For example in a picture of a flower, some parts of the picture may represent  petals while others represent the stem.
}

\longnewglossaryentry{inaccessible}{
        name={inaccessible}
      }{An attribute is \emph{inaccessible} in a context just if it is difficult or impossible,
in that context, to discern substantive truths about that attribute.  For example,
in ordinary life and for most people the attribute \emph{being further from Kilmery (in Wales) than
Steve’s brother Matt is} would be inaccessible.

See \citet[p.~271]{kahneman:2005_model}: ‘We adopt the term accessibility to refer to the ease (or effort) with which particular mental contents come to mind.’
}

\longnewglossaryentry{individual foundations}{
        name={individual foundations}
      }{Categories of moral concern linked to individual needs; these are often taken to be
harm/care, cheating/fairness \citep{graham:2011_mapping}. Sometimes called \emph{individualizing foundations}.
}

\longnewglossaryentry{inferential integration}{
        name={inferential integration}
      }{For states to be \emph{inferentially integrated} means that: (a) they can come to be nonaccidentally 
related in ways that are approximately rational thanks to processes of inference and practical reasoning;
and
(b) in the absence of obstacles such as time pressure, distraction, motivations to be 
irrational, self-deception or exhaustion, approximately rational harmony will 
characteristically be maintained among those states that are currently active.
}

\longnewglossaryentry{informational encapsulation}{
        name={informational encapsulation}
      }{One process is informationally encapsulated from some other processes to the extent that
there are limits on the one process’ ability to consume information available to the other processes.
(See \citealp{Fodor:1983dg}; \citealp[pp.~5ff]{clarke:2020_cognitive}.)
}

\longnewglossaryentry{innate}{
        name={innate}
      }{Not learned.
While everyone disagrees about what innateness is (see
\citealp{Samuels:2004ho}), on this course a cognitive ability is innate just
if its developmental emergence is not a direct consequence of data-driven
learning.
}

\longnewglossaryentry{instrumental action}{
        name={instrumental action}
      }{An action is \emph{instrumental} if it happens in 
order to bring about an outcome,
as when you press a lever in order to obtain food. (In this case,
obtaining food is the outcome, lever pressing is the action, and the
action is instrumental because it occurs in order to bring it about
that you obtain food.)  

Note that you may encounter other definitions of instrumental in the literature.
\citet[p.~177]{dickinson:2016_instrumental}
characterises instrumental actions differently: in place of the teleological
‘in order to bring about an outcome’, he stipulates that an instrumental
action is one that is ‘controlled by the contingency between’ the action
and an outcome. And \citet[p.~464]{dewit:2009_associative} stipulate that
‘instrumental actions are *learned*’. 
}

\longnewglossaryentry{intentional isolation}{
        name={intentional isolation}
      }{Two representations are \emph{intentionally isolated} when the only 
links between them, if any, are provided by \glspl{intentional isolator}.
}

\longnewglossaryentry{intentional isolator}{
        name={intentional isolator}
      }{An event or state which links representations but either lacks intentional features entirely or else has intentional features that are only very distantly related to those of the two representations
it links.
\Glspl{metacognitive feeling} and behaviours are paradigm intentional isolators.
}

\longnewglossaryentry{loose reconstruction}{
        name={loose reconstruction}
      }{(of an argument). A reconstruction which prioritises finding a correct argument
for a significant conclusion over faithfully representing the argument being reconstructed.
}

\longnewglossaryentry{metacognitive feeling}{
        name={metacognitive feeling}
      }{A metacognitive feeling is a feeling which is caused by
a \gls{metacognitive process}.
Paradigm examples of metacognitive feelings include the feeling of
familiarity, the feeling that something is on the tip of your tongue, the
feeling of confidence and the feeling that someone’s eyes are boring into
your back. 
On this course, we assume that one characteristic of metacogntive
feelings is that either they lack intentional objects altogether, or else
what their subjects take them to be about is typically only very distantly
related to their intentional objects. (This is controversial---see
\citealp{dokic:2012_seeds} for a variety of conflicting theories.)
}

\longnewglossaryentry{metacognitive process}{
        name={metacognitive process}
      }{A process which monitors another cognitive process.
For instance, a process which monitors the fluency of recall, or of action selection,
is a metacognitive process.
}

\longnewglossaryentry{mindreading}{
        name={mindreading}
      }{The process of identifying a mental state as a mental state that some particular individual, another or yourself, has.
To say someone has a \emph{theory of mind} is another way of saying that she is capable of mindreading.

According to an influential definition offered by \citet[p.~515]{premack_does_1978},
for an individual to have a theory of mind its for her to ‘impute mental states to himself \emph{and} to others’ (my italics).
(I have slightly relaxed their definition by changing their ‘and’ to ‘or’ in order to allow for the possibility that there are mindreaders who can identify others’ but not their own mental states.)
}

\longnewglossaryentry{minimal model of minds and actions}{
        name={minimal model of minds and actions}
      }{A model specified by a \gls{minimal theory of mind}.
}

\longnewglossaryentry{minimal theory of mind}{
        name={minimal theory of mind}
      }{A theory of the mental in which: (a) mental states are assigned functional
roles that can be readily codified; and, (b), the contents of mental states
can be distinguished by things which, like locations, shapes and colours,
can be held in mind using some kind of quality space or feature map.
}

\longnewglossaryentry{model}{
        name={model}
      }{A model is a way some part or aspect of the world could be.
}

\longnewglossaryentry{model based}{
        name={model based}
      }{A \emph{model-based} process is one that relies on a \gls{model}.  This is usually thought
to involve deriving predictions from representations of the model.
Compare \citet[p.~477]{dayan:2014_modelbased}: ‘A model-based strategy involves prospective 
cognition, formulating and
pursuing explicit possible future scenarios based on internal representations
of stimuli, situations, and environmental circumstances.’
}

\longnewglossaryentry{model free}{
        name={model free}
      }{A \emph{model-free} process is one that does not rely on a \gls{model}.  This term is often
used for processes which exploit causal or statistical connections that are not represented.
}

\longnewglossaryentry{model of minds and actions}{
        name={model of minds and actions}
      }{A model of minds and actions is a way mental aspects of the world could be.
A model is not a theory, although it may be specified by one.
}

\longnewglossaryentry{module}{
        name={module}
      }{A \emph{module} is standardly characterised as a cognitive system which exhibits, to a significant degree,
a set of features including \glslink{domain specific}{domain specificity}, 
limited accessibility, and \gls{informational encapsulation}.
Contemporary interest in modularity stems from \citet{Fodor:1983dg}.  Note that there are now a wide 
range of incompatible views on what modules are and little agreement among researchers on what
modules are or even which features are characteristic of them.
}

\longnewglossaryentry{moral conviction}{
        name={moral conviction}
      }{‘Moral conviction refers to a strong and absolute belief that something is right or wrong, moral or immoral’ \citep[p.~896]{skitka:2005_moral}.
}

\longnewglossaryentry{moral disengagement}{
        name={moral disengagement}
      }{Moral disengagement occurs when self-sanctions are disengaged from inhumane
conduct. \citet[p.~103]{bandura:2002_selective} identifies several
mechanisms of moral disengagement: ‘The disengagement may centre on
redefining harmful conduct as honourable by moral justification, exonerating
social comparison and sanitising language. It may focus on agency of action
so that perpetrators can minimise their role in causing harm by diffusion
and displacement of responsibility. It may involve minimising or distorting
the harm that follows from detrimental actions; and the disengagement may
include dehumanising and blaming the victims of the maltreatment.’
}

\longnewglossaryentry{moral dumbfounding}{
        name={moral dumbfounding}
      }{‘the stubborn and puzzled maintenance of an [ethical] judgment without supporting reasons’ \citep[p.~1]{haidt:2000_moral}.}

\longnewglossaryentry{Moral Foundations Theory}{
        name={Moral Foundations Theory}
      }{The theory that moral pluralism is true; moral foundations are innate but also subject to 
cultural learning, and the \gls{Social Intuitionist Model of Moral Judgement} is correct \citep{graham:2019_moral}.
Proponents often claim, further, that cultural variation in how these innate foundations 
are woven into ethical abilities 
can be measured using the Moral Foundations Questionnare
(\citealp{graham:2009_liberals}; \citealp{graham:2011_mapping}).
Some empirical objections have been offered (\citealp{davis:2016_moral}; \citealp{davis:2017_purity}; \citealp{dogruyol:2019_fivefactor}).
See \ref{unit:moral_foundations_theory}.
}

\longnewglossaryentry{moral intuition}{
        name={moral intuition}
      }{According to this lecturer, moral intuitions are unreflective ethical judgements.

According to \citet[p.~256]{sinnott:2010_moral}, moral intuitions are ‘strong, stable, immediate moral beliefs.’
}

\longnewglossaryentry{moral psychology}{
        name={moral psychology}
      }{The study of ethical abilities. These include abilities to act in accordance with ethical considerations, to make ethical judgments, to exercise moral suasion, and to feel things in response to unethical or superordinate acts.}

\longnewglossaryentry{moral reframing}{
        name={moral reframing}
      }{’A technique in which a position an individual would not normally support is 
framed in a way that it is consistent with that individual's moral values.
[...]  In the political arena, moral reframing involves arguing in favor of a 
political position that members of a political group would not normally support 
in terms of moral concerns that the members strongly ascribe to‘
\citep[pp.~2--3]{feinberg:2019_moral}.
}

\longnewglossaryentry{moral sense}{
        name={moral sense}
      }{A ‘tendency to see certain actions and individuals as right, good, and
deserving of reward, and others as wrong, bad, and deserving of punishment’ \citep[p.~186]{hamlin:2013_moral}.
}

\longnewglossaryentry{motor representation}{
        name={motor representation}
      }{The kind of representation characteristically involved in preparing, performing and monitoring sequences of small-scale actions such as grasping, transporting and placing an object.
They represent actual, possible, imagined or observed actions and their effects. 
}

\longnewglossaryentry{Motor Theory of Goal Tracking}{
        name={Motor Theory of Goal Tracking}
      }{Tracking goals is acting in reverse 
\citep{sinigaglia:2015_goal_ascription}.
}

\longnewglossaryentry{Near Alone}{
        name={Near Alone}
      }{‘I am walking past a pond in a foreign country that I am
visiting. I alone see many children drowning in it, and I alone can save one
of them.
To save the one, I must put the 500 dollars I have in my pocket into a machine
that then triggers (via electric current) rescue machinery that will certainly
scoop him out’
\citep[p.~348]{kamm:2008_intricate}
}

\longnewglossaryentry{non A task}{
        name={non A task}
      }{Any false belief task that typically developing children tend to pass in
their first or second year of life.
}

\longnewglossaryentry{not-justified-inferentially}{
        name={not-justified-inferentially}
      }{A claim (or premise, or principle) is not-justified-inferentially if it is not
justified in virtue of being inferred from some other claim (or premise, or principle). 

Claims made on the basis of perception (\emph{That jumper is red}, say) are typically
not-justified-inferentially.

Why not just say ‘noninferentially justified’?  Because that can be read as implying that 
the claim \emph{is} justified, noninferentially. Whereas ‘not-justified-inferentially’ does not imply this.
Any claim which is not justified at all is thereby not-justified-inferentially.
}

\longnewglossaryentry{object permanence}{
        name={object permanence}
      }{the ability to track objects while briefly unperceived.}

\longnewglossaryentry{outcome}{
        name={outcome}
      }{An outcome of an action is a possible or actual state of affairs.
}

\longnewglossaryentry{perceptual animacy}{
        name={perceptual animacy}
      }{The detection by broadly perceptual processes of animate objects \citep[pp.~201--2]{scholl:2013_perceiving}.
}

\longnewglossaryentry{poverty of stimulus argument}{
        name={poverty of stimulus argument}
      }{An argument used to establish that something is \gls{innate}.
A poverty of stimulus argument aims to establish that something humans acquire is not acquired by data driven-learning (see \citealp{pullum:2002_empirical}).
}

\longnewglossaryentry{poverty of theory argument}{
        name={poverty of theory argument}
      }{An inexpensive alternative to a poverty of stimulus argument.
}

\longnewglossaryentry{Principles of Object Perception}{
        name={Principles of Object Perception}
      }{These are thought to include no action at a distance, rigidity, boundedness and cohesion.
}

\longnewglossaryentry{problem}{
        name={problem}
      }{a question that is difficult to answer.
}

\longnewglossaryentry{proximal outcome}{
        name={proximal outcome}
      }{The \glspl{outcome} of an action can be partially ordered by the cause-effect
relation. For one outcome to be more \emph{proximal} than another is for it to
be higher with respect to that partial ordering. To illustrate, if you kick
a ball through a window, the kicking is a more proximal outcome than
the window’s breaking.
}

\longnewglossaryentry{pure goal-tracking}{
        name={pure goal-tracking}
      }{Tracking goals is \emph{pure} when does not involve ascribing intentions or any other mental states.
}

\longnewglossaryentry{purposive action}{
        name={purposive action}
      }{an action that is directed to bringing about an outcome,
such as pressing a lever in order to obtain food. (In this case,
obtaining food is the outcome, lever pressing is the action, and the
action is purposive because it is directed to the outcome.)
}

\longnewglossaryentry{reflective equilibrium}{
        name={reflective equilibrium}
      }{A project which aims to provide a set of general principles which cohere with 
the judgements you are, on reflection, inclined to make about particular cases in this sense:
the principles ‘when conjoined to our beliefs and knowledge of the circumstances, would 
lead us to make thse judgemnts with their supporting reasons were we to apply these principles’ \citep[p.~41]{rawls:1999_theory}. For background, see \citet{daniels:2003_reflective}.
}

\longnewglossaryentry{replicate}{
        name={replicate}
      }{To \emph{replicate} a experiment is to attempt to repeat it with the aim of reproducing the original findings.
Where the original findings are not found, it is called a \emph{failed replication}.

A replication can be more or less \emph{direct}; that is, it may adhere very closely to the 
original experiment, or it may include varations in the stimuli, subjects and settings.
Very indirect replications are sometimes called \emph{conceptual replications}.
}

\longnewglossaryentry{representational format}{
        name={representational format}
      }{Format is an aspect of representation distinct from content (and from vehicle). Consider that
a line on a map and a list of verbal instructions can both represent the same route through a city.
They differ in format: one is cartographic, the other linguistic. 
}

\longnewglossaryentry{representational momentum}{
        name={representational momentum}
      }{Sometimes when adult humans observe a moving object that disappears, they
will misremember the location of its disappearance in way that reflects its
momentum \citep{freyd:1984_representational,hubbard:2010_rm}.
There are several competing models of representational momentum
and related phenomena involving misremembered location
\citep{hubbard:2010_rm}.
}

\longnewglossaryentry{representations and algorithms}{
        name={representations and algorithms}
      }{To specify the representations and algorithms involved in a system is to specify how the inputs and outputs are represented and how the transformation from input to output is accomplished.
\citet{Marr:1982kx} distinguishes the {representations and algorithms}  from the  \gls{computational description} of a system and its hardware implementation.
}

\longnewglossaryentry{self-evident}{
        name={self-evident}
      }{‘self-evident propositions
are truths meeting two conditions: (1) in 
virtue of adequately understanding them, one has justification for believing 
them [...]; and (2) believing them on the basis of adequately 
understanding them entails knowing them’ \citep[p.~65]{audi:2015_intuition}.
}

\longnewglossaryentry{signature limit}{
        name={signature limit}
      }{A signature limit of a system is a pattern of behaviour the system exhibits which is both defective given what the system is for and peculiar to that system.
A {signature limit} of a model is a set of predictions derivable from the model which
are incorrect, and which are not predictions of other models under consideration.
}

\longnewglossaryentry{Simple View}{
        name={Simple View}
      }{This term is used for two thematically related claims.
Concerning physical objects, the Simple View is the claim that the \gls{Principles of Object Perception} are things we know or believe, and we generate expectations from these principles by a process of inference.
Concerning the goals of actions, the Simple View is the claim that
the principles comprising the Teleological Stance are things we know or believe, and we are able to \glspl{track a goal} by making inferences from these principles.
}

\longnewglossaryentry{slow}{
        name={slow}
      }{converse of \gls{fast}.
}

\longnewglossaryentry{Social Intuitionist Model of Moral Judgement}{
        name={Social Intuitionist Model of Moral Judgement}
      }{A model on which intuitive processes are directly responsible for moral judgements \citep{haidt:2008_social}.
One’s own reasoning does not typically affect one’s own moral judgements,
but (outside philosophy, perhaps) is typically used only to provide post-hoc justification 
after moral judgements are made.
Reasoning does affect others’ moral intuitions, and so provides a mechanism for cultural learning.
}

\longnewglossaryentry{stimulus}{
        name={stimulus}
      }{A \emph{stiumlus} is just a situation or event. Typically, ‘stimlus’ is used to label
things which do, or might, prompt actions such as the presence of a lever or the 
flashing of a light.
}

\longnewglossaryentry{target}{
        name={target}
      }{The \emph{target} or \emph{targets} of an action (if any) are the things the towards which it is directed.}

\longnewglossaryentry{Teleological Stance}{
        name={Teleological Stance}
      }{To adopt the Teleological Stance is to exploit certain principles concerning the optimality of 
goal-directed actions in tracking goals \citep{Csibra:1998cx}.
}

\longnewglossaryentry{track}{
        name={track}
      }{For a process to \emph{track} an attribute is for the presence or absence of the attribute 
to make a difference to how the process unfolds, 
where this is not an accident. (And for a system or device to track an attribute is for some process
in that system or device to track it.)

Tracking an attribute is contrasted with \emph{computing} it.
Unlike tracking, computing typically requires that the attribute be represented.
}

\longnewglossaryentry{track a goal}{
        name={track a goal}
      }{For a process to track a goal of an action is for how that process unfolds to nonaccidentally depend in some way on whether that outcome is indeed a goal of the action.
For someone to track the goals of an action is for  there to be processes in her which track one or more goals of that action.
}

\longnewglossaryentry{Transplant}{
        name={Transplant}
      }{A dilemma. Five people are going to die but you can save them all by 
cutting up one healthy person and distributing her organs.
Is it ok to cut her up?
}

\longnewglossaryentry{Trolley}{
        name={Trolley}
      }{A dilemma; also known as \emph{Switch}. A runaway trolley is about to run over 
and kill five people. 
You can hit a switch that will divert the trolley onto a different set of tracks 
where it will kill only one. 
Is it okay to hit the switch? 
}

\longnewglossaryentry{trolley cases}{
        name={trolley cases}
      }{Scenarios designed to elicit puzzling or informative patterns of judgement about 
how someone should act. Examples include \gls{Trolley}, \gls{Transplant}, and \gls{Drop}. 
Their use was pioneered by \citet{foot:1967_problem} and 
\citet{thomson:1976_killing}, who aimed to use them to understand ethical considerations
around abortion and euthanasia. 
}

\longnewglossaryentry{trolley problem}{
        name={trolley problem}
      }{‘Why is it that \gls{Edward} may turn that trolley to save his five, but \gls{David}
may not cut up his healthy specimen to save his five?’
\citep[p.~206]{thomson:1976_killing}.
}

\longnewglossaryentry{unfamiliar problem}{
        name={unfamiliar problem}
      }{An unfamiliar problem (or situation) is one ’with which we have inadequate evolutionary, cultural, or personal experience’ \citep[p.~714]{greene:2014_pointandshoot}.}

\longnewglossaryentry{useful construct}{
        name={useful construct}
      }{A \emph{useful} construct is one that can explain an interesting range of target phenomena.
}

\longnewglossaryentry{valid construct}{
        name={valid construct}
      }{For the purposes of this course, a \emph{valid} construct is one that can be measured using a tool (often a questionnaire)
where there is sufficient evidence to conclude that the tool measures the construct.
When used for cross-cultural comparisons, the tool should exhibit metric and scalar 
invariance (i.e. it should measure the same construct in the same way irrespective of which
the culture participant belongs to). 

Note that the term ‘construct validity’ is used in many different ways.  It is probably best
to try to understand it case-by-case---each time the term occurs, ask yourself what the researchers
are claiming to have shown. If you do want an overview, \citet{drost:2011_validity} is one source.
}

\longnewglossaryentry{violation-of-expectation}{
        name={violation-of-expectation}
      }{Violation-of-expectation experiments test hypotheses about what infants
expect by comparing their responses to two events.
The responses compared are usually looking durations.
Looking durations are linked to infants’ expectations by the assumption
that, all things being equal, infants will typically look longer at
something which violates an expectation of theirs than something which
does not.
Accordingly, with careful controls, it is sometimes possible to draw
conclusions about infants’ expectations from evidence that they generally
look longer at one event than another.
}