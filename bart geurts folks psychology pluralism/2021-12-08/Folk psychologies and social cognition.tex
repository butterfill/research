\documentclass[12pt]{article}
% \usepackage{gooseberry}
%\usepackage{enumitem,wasysym,titlesec,graphicx}
\usepackage[english]{babel}\frenchspacing
%\usepackage[alwaysadjust]{paralist}
%\newcommand{\E}[1]{\emph{#1}}
%\newcommand{\Q}[1]{``#1''}
%%\newcommand{\K}[1]{\textsc{#1}}
%\newcommand{\hs}[1]{\hspace*{#1ex}}
%\newcommand{\vs}[1]{\vspace{#1ex}}
%\renewenvironment{enumerate}[1]{\pltopsep = 0.6ex\begin{compactenum}[#1]}{\end{compactenum}}
%\renewenvironment{itemize}{\begin{enumerate}{$\circ$}}{\end{enumerate}}
%\titleformat{\section}{\vs1}{\large\thesection.}{1ex}{\large}%{\itshape}
%\titleformat{\subsection}{\vs0\normalsize}{\thesubsection.}{1ex}{}

\title{Folk psychologies and social cognition}
\author{Bart Geurts \\ Stephen Butterfill}
\date{}

\begin{document}

\maketitle

%\thispagestyle{firstpage}
%----------------------------------------------
\section{Introduction}
%----------------------------------------------
The word \Q{folk psychology} (and related words such as \Q{mindreading}, \Q{theory of mind}, etc.) has been in many different ways. Not necessarily a bad thing, but can be confusing. Three main uses, which are related (\citeNameNP{Laanpere:2016}, \citeNameNP{Molder:2016}):
\begin{enumerate}{\small 1.}
\item FP as a \E{practice} of attributing mental states. Attribution may be overt or covert (private).
\item FP as the \E{ability} to understand or at least track mental states of others \cite{Butterfill:2019}.
\item FP as a \E{framework} of mentalist concepts, which may be like a theory in some respects.
\end{enumerate}
These uses are related. If you endorse more than one, you're treating \Q{folk psychology}  as a polysemous word.\vs1

\noindent Issues:
\begin{itemize}
\item The first sense draws together (at least) two uses that are fundamentally different: psychological (covert/private) vs. social/communicative (overt). 
\item Moreover, the former is often viewed as underpinning the former, which may be true, but is hardly a theory-neutral assumption.
\item Underlying these two practices are abilities that may overlap, but are clearly distinct.
\item Our proposal incorporates these uses, but redraws the general picture. Key dichotomy: social practices (SPs) vs. social cognition (SC). Will propose subdivisions within each domain.
\item It is not our ambition to \E{define} terms like \Q{folk psychology}, \Q{social practice}, and so on. More like clearing the terrain by trying to make tentative distinctions that look promising; more promising, at any rate, than the distinctions that have been made so far. (When Newton began to ponder why things always fall downwards and in a straight line, his first order of business was not to define what that might mean; the question was sufficiently clear for the kind of pondering that needed to be done. The questions that have thus far been raised about folk psychology and its kin haven't reached that stage yet.)
\end{itemize}

%----------------------------------------------
\section{Social practices}
%----------------------------------------------
Within the domain of social practices, at least three types may be distinguished:\vs{-1}
%----------------------------------------------
\subsection*{\normalfont{SP}$_1:$ Communicative practices involving such locutions as:}
%----------------------------------------------
\parbox[c]{.33\linewidth}{\small\raggedright
\begin{enumerate}{\hspace*{.5ex}}\itemsep=.2ex
\item I \E{like} you.
\item I don't \E{mind}.
\item I  \E{feel bad}.
\item I \E{forgot} to do it.
\item I lost my \E{sense of smell}.
\item I wasn't \E{expecting} this.
\item I did it against my \E{will}.
\end{enumerate}}\hspace{.5ex}
\parbox[c]{.33\linewidth}{\small\raggedright
\begin{enumerate}{\hspace*{.5ex}}\itemsep=.2ex
\item Why are you \E{angry}?
\item What do you \E{mean}?
%\item Do you really \E{want} that?
%\item Don't you \E{remember}?
\item \E{Think} about it.
\item \E{Look!}
%\item \E{Listen!}
\item You're \E{stupid}.
\item You don't \E{understand}.
\item You don't \E{love} me.
\end{enumerate}}\hspace{.5ex}
\parbox[c]{.33\linewidth}{\small\raggedright
\begin{enumerate}{\hspace*{.5ex}}\itemsep=.2ex
%\item They seem \E{nervous}.
\item He did it \E{on purpose}.
\item She's \E{considering} it.
\item They \E{agree} on that.
\item They are one \E{mind}.
%\item That's his \E{wish}.
\item Her \E{soul} is pure.
\item He \E{suspects} it will rain.
%\item She \E{saw} a mouse.
\item Her \E{memory} is fading.
\end{enumerate}}\vs1

\begin{itemize}
\item SP$_1$s are thoroughly interactive \cite{Clark:1996}. Speech acts require addressees and addressees need to be responsive. Uptake, agreement, backchannel cues. 
\item There are important differences between SP$_1$s associated with different languages/cultures \cite{Lillard:1998}, and even between populations sharing the same language. 
\end{itemize}
%----------------------------------------------
\subsection*{\normalfont{SP}$_2:$ Non-communicative practices involving mentalist locutions}
%----------------------------------------------
\begin{enumerate}{--}\itemsep=.2ex
\item A says to herself: \Q{He seems angry.}
\item B asks himself: \Q{Why am I so nervous?}
\end{enumerate}

\begin{itemize}
\item Non-social talk derives from social talk. SP$_2$ derives from SP$_1$, but is obviously not interactive in the same strict sense.
\item There are many forms and uses of non-social talk: self-addressed speech acts (as in these examples), rehearsing speech, memorisation, etc. (\citeNameNP{Winsler:2009a}, \citeNameNP{Gregory:2016}, \citeNameNP{Geurts:2018})
\item Non-social uses of language need not be accessible to consciousness.
\item Some of our SP$_1$-practices are \E{about} SP$_2$-practices. E.g., \Q{A said to herself that B seemed angry.}
\item Mental/non-mental divide problematic.
\end{itemize}

%----------------------------------------------
\subsection*{\normalfont{SP}$_3:$ Non-communicative practices that don't involve mentalist locutions:}
%----------------------------------------------
\begin{enumerate}{--}\itemsep=.2ex
\item B is crying and A is trying to comfort him.
\item A sees B approaching with a gloomy look on his face, and tries to avoid him.
\item Approaching each other in a narrow corridor, A steps to the right, and B does too.
\end{enumerate}
\begin{itemize}
\item This family of social practices is hard if not impossible to characterise in positive terms.
\item In each of these cases, it is natural to represent B's actions in SP$_1$-terms, regardless whether these terms figure in B's cognition or not.
\item Whereas SP$_1$ and SP$_2$ are language-dependent and the prerogative of our species, at least some SP$_3$s occur in other species and may be genetically determined to a large degree. 
\item Examples: dogs, gaze following, attribution of basic emotions \cite{Goldie:2002}
\end{itemize}

%----------------------------------------------
\section{\Q{Frameworks}}
%----------------------------------------------
MTM \cite{Butterfill:2013}, \citeName{Geurts:2021}

%----------------------------------------------
\section{Social cognition}
%----------------------------------------------

Abilities vs. \Q{mechanisms}.

%\noindent Research questions: 
%\begin{itemize}
%\item What distinguishes interactions that involve such locutions from interactions that don't? 
%\item What purposes do they serve? 
%\item How do such interactions vary between languages, cultures, and other social groups? \item How do children acquire this style of interaction?
%\item How did it evolve?
%\end{itemize}
%
%Lewisian platitudes could be accommodated, too, provided they are not too abstruse. Although people rarely bother to affirm that toothaches are pains, they will readily agree when asked, and therefore this type of interaction would be on our list, too. But we are primarily concerned with interactions, i.e. language \E{use}, rather than any propositions they may involve.
%
%Folk-psychological interactions aren't random. They have a large measure of systematicity, \E{some} of which may be theory-like (ontology, principles,\ldots). But a folk psychology is not just a model/picture of (part of) the world---though it may be that, too. FPs are normative systems in various respects (Brandom, McGeer, Zawidzki, Laanpere). Examples: there are quite a few things one is supposed to believe; self-attribution of some feelings may be virtually mandatory in some situations (\Q{I'm sorry}); there are cultures in which mental-state talk is discouraged/proscribed in some/many situations (Robbins). (More about normativity later.)
%
%We will generally use the term \Q{folk psychology} extensionally, to refer to certain patterns of social interaction, and sometimes intensionally, to refer to the systematicity underlying such patterns.
%
%A folk psychology is a social practice, the bulk of which is acquired as part of a language. (There are non-verbal ways of attributing mental states, like tapping one's index finger against one's head to express that a person is non compos mentis.) It follows from this that pre-linguistic children and non-human animals don't engage in FP as we use that term. 
%
%Pre-linguistic children and many non-human animals do have social cognition: a set of psychological capacities that enables their owners to engage in social interactions. FP requires advanced forms of social cognition. In an evolutionary perspective, while social cognition is an ancient phenomenon, folk psychology is a recent development. 
%
%Important point: strict separation between social and psychological domains/levels: FP is part of the social domain; social cognition is part of our psychology.
% 
%%----------------------------------------------
%\section{Common features of mental states}
%%----------------------------------------------
%\begin{itemize}
%\item The paradigmatic instances folk-psychological speech acts are assertions that serve to attribute a mental state to a third party; belief attribution has been discussed most. Our list contains several examples (third column), but also shows that there is a lot more: questions, orders, requests, etc. Useful reminder that mental-state attribution is a form special. E.g., when I ask you, \Q{Do you really believe that?}\hs{-.5}, I patently use the notion of belief without attributing it to anybody.
%\item In this section we list a number of features that, amongst speakers of English, are commonly associated with mental states, using belief and belief attribution as our running examples.
%\end{itemize}
%
%\begin{enumerate}{\hs2\small1.}
%
%\item\label{p:content} Many though not all of the mental states that we attribute to one another appear to have propositional content: descriptive content that either agrees with the facts or not.
%\par Mental states that needn't have propositional content, thus understood, are nervousness, anger,\ldots
%\item \label{p:effects} Mental states guide our actions, and there are more or less systematic connections between mental states, on the one hand, and patterns of behaviour, on the other. 
%\par Corollary: things with mental states are things that act. 
%\par Mental states give us reason to act, justify our actions, etc. [normativity]
%
%\item \label{p:causes} Mental-state attribution (or the mental states themselves?) obeys certain rules/regularities: seeing is believing, inertia of belief, elementary logic, interactions between mental states (e.g. belief and intention),\ldots
%\par Part of this systematicity involves content. 
%\par Normativity.
%
%\item Beliefs are persistent (cf. Bratman on intention). Persistence comes in at least two flavours. If you believe at noon that it is raining, then ceteris paribus you will believe tonight that it was raining \E{at noon}; but you won't necessarily believe tonight, even ceteris paribus, that it is \E{still} raining. (Compare this with your belief that Angela Merkel is German.) So, once you have formed a belief, you will stick with it, ceteris paribus; this is persistence proper. The other is that, based on world knowledge, you suppose that certain states of the world are inert, ceteris paribus, and your beliefs reflect these inertia assumptions. 
%
%\item \label{p:privacy} Mental states are private.
%
%\item Gradability
%
%\item \label{p:location} Mental states are somewhere. In our culture they are generally between the ears. Exceptions are pains, itches,\ldots
%
%\end{enumerate}
%
%%----------------------------------------------
%\section{Social cognition}
%%----------------------------------------------
%\begin{itemize}
%\item Social cognition concerns the psychological capacities or processes that are specifically involved in social interactions between conspecifics. As applied to humans it covers a quite significant portion of our psychology, but here we are primarily interested in those aspects of social cognition that are involved in, or at least related to, our folk-psychological interactions.%\Q{Social cognition concerns the various psychological processes that enable individuals to take advantage of being part of a social group.} (Frith 2008)
%\item Examples:
%\begin{enumerate}{\small1.}
%\item distinguishing between purposive and accidental behaviours
%\item tracking eye gaze
%\item recognising basic emotions
%\item registration
%\end{enumerate}
%\end{itemize}
%
%%----------------------------------------------
%\section{Folk psychology and social cognition}
%%----------------------------------------------
%How do folk psychology and social cognition relate to one another? This is the hardest question, and we don't have anything like a complete answer. But still, \ldots\vs1
%
%\noindent General aproach:
%\begin{itemize}
%\item The question concerns relations between two fundamentally different domains: interpersonal interaction and intrapersonal psychology. These can be addressed separately.
%\item We adopt an evolutionary perspective, in which a new developments in the social domain (folk psychology) build on and modify capacities in the psychological domain (social cognition).
%\item Instead of offering a general theory up front, we will begin by looking at a number of concrete cases first.
%\end{itemize}
%
%\subsection{Acting on purpose}
%Barney has dropped Betty's Qing vase on the kitchen floor, as a consequence of which the beloved vessel has ceased to be priceless. How will Betty react? That depends a great deal on whether Barney acted on purpose or not, and although it is hard if not impossible to say precisely how we do it, most of us would agree that, in a great many cases, it can be \E{seen} that an object is being dropped deliberately, and it seems plausible to suppose that,  between humans, there is substantial inter-observer agreement on the distinction between deliberate and non-deliberate behaviours.
%
%\begin{itemize}
%\item Many species have been reported to respond differentially to deliberate and non-deliberate behaviours, so we are not alone. However, this way of describing a cross-species pattern of behaviours is liable to mislead, because \Q{deliberate} and \Q{non-deliberate} are folk-psychological terms, which non-human animals don't have. Therefore, let's use the term  \Q{d-behaviours} as a neutral designation of our target phenomenon, which in English is associated with such expressions as \Q{deliberate}, \Q{on purpose}, and so on.
%\item At some point in prehistoric time, our ancestors had started to respond differentially to d-behaviours (social domain) and therefore had the capacity to distinguish such behaviours (social cognition). This part of the story is unproblematic, even if it remains to be fleshed out.
%\item Then a new category of folk-psychological terms began to appear: people began saying things like, \Q{I didn't do it on purpose.} The notion of doing something on purpose was linked to d-behavioural patterns. Dropping a vase in a d-manner counted as evidence that the vase was dropped on purpose. But \Q{on purpose} was not just a label for d-behaviours, for it became associated with variety of communicative practices, notably normative ones. For example, it became associated with moral notions like responsibility and blameworthiness: if Barney  dropped Betty's vase in a d-kind of way, her response would be called \E{justified} anger rather than mere anger or exasperation.
%\begin{center}\small
%\parbox[c]{19ex}{\raggedleft Social interaction:}\hs1  
%\fbox{\footnotesize\parbox[c]{23ex}{\centering responding differentially to d-behaviours\rule[-.5ex]{0pt}{2ex}}}{{\ \large$\sim$}}
%\fbox{\footnotesize\parbox[l]{23ex}{\centering using and responding to  locutions like \Q{on purpose}}}\vs{.5}
% 
%\parbox[c]{19ex}{\hs1}\hs1
%\parbox[c]{25ex}{\centering\rotatebox{90}{\large$\sim$}}\hs3
%\parbox[c]{25ex}{\centering\rotatebox{90}{\large$\sim$}}\vs{.5}
%
%\parbox[c]{19ex}{\raggedleft Social cognition:}\hs1
%\fbox{\footnotesize\parbox[c]{23ex}{\centering capacity to respond differentially to d-behaviours\rule[-.5ex]{0pt}{2ex}}}{{\ \large$\sim$}}
%\fbox{\footnotesize\parbox[c]{23ex}{\centering capacity to use and respond to locutions  like \Q{on purpose}}}
%\end{center}
%\vs1
%\item  The tildes mark relations that require an explanation. Exchanging locutions like \Q{on purpose} is part of our folk psychology; the capacities in the bottom row are part of our social cognition. Since the \Q{logic of discovery} proceeds from observations in the social domain to inferences about the psychological domain, we'll start our discussion of boxes and relations in the top right-hand corner.
%
%\begin{enumerate}{1.}
%
%\item There is ample evidence that speakers use terms like \Q{on purpose}, \Q{deliberately}, \Q{intentionally},  that they talk about and inquire into each other's motives, and so on. All this is readily observable in a great many situations, and if more data are needed, street interviews may be conducted, questionnaires may be used, etc. This is the kind of thing that ethnographers, lexicographers, and language philosophers have been doing, and it's fairly straightforward. Charting the underlying systematicity is more challenging, but much worse is  to come when we turn to the other boxes.
%
%\item In general, it is much harder to determine whether, and if so how, the members of a population respond differentially to one another's d-behaviours. Often requires careful experimentation. Differences between species. Fairly clear why these behavioural patterns evolved: others' d-behaviours tend to be more relevant to my interests than their non-d-behaviours.
%\vs1\par Relations between 1 and 2:
%    \begin{itemize}
%    \item 2 is co-opted into 1: we \E{label} d-behaviours as \Q{deliberate}, \Q{on purpose}, etc.,  thereby \E{interpreting} them in these terms by linking them to shared evaluative attitudes and practices.
%    \item By the same token, our purposive vocabulary is \E{grounded} in pre-linguistic patterns of social behaviour.
%    \end{itemize}
%
%\item Turning to the cognition underlying 2, things become significantly murkier. 
%    \begin{itemize}
%    \item Presumably perception plays a key role.
%    \item In other species than ours, the capacity to respond differentially to d-behaviours may be fully hard-wired, and I guess that in humans it is at least partly hard-wired.
%    \item But it may also be shaped in part by \Q{higher-order} factors. Routinisation. 
%    \end{itemize}
%    
%\item Mystery world
%\end{enumerate}
%\end{itemize}
%%----------------------------------------------
%\newpage\section*{Appendix: Thinking things}
%%----------------------------------------------
%\begin{enumerate}{\small1.}\itemsep=0ex
%\item \Q{my husband thinks}: 610.000 hits \\ \Q{My husband thinks he is always right.}
%\item \Q{my dog thinks}: 363.000 hits \\ \Q{My dog thinks I'm a genius.}
%\item \Q{my phone thinks}: 121,000 hits (cf. \Q{my phone says}: 777.000 hits) \\ \Q{My phone thinks I'm in Canada.}
%\item \Q{my cat thinks}: 108,000 hits \\ \Q{My cat thinks he's human.}
%\item \Q{the bees think}: 57,100 hits \\ \Q{The bees think it's June.}
%\item \Q{the thermostat thinks}: 4,600 hits \\ \Q{With the module set to the middle setting, the actual room temperature would be about 62 degrees while the thermostat thinks it is 72 degrees.}
%\item \Q{my calendar thinks}: 300 hits (cf. \Q{my calendar says}: 147.000 hits)\\\Q{My calendar thinks it is May 4th, how do I reset it to May 9th?}
%\item \Q{my pencil thinks}: 50 hits \\\Q{Too bad my pencil thinks that it's a helicopter and keeps flying away.}
%\end{enumerate}
%
%\begin{center}\small
%\begin{tabular}{r@{ }l@{ \ }l@{ \ }l@{ \ }l@{ \ }l}\hline
%& ...\/knows & ...\/thinks & ...\/believes & ...\/wants & ...\/intends \\\hline
%he\/... &\ \ 127905 &\ \ 78569 &\ \ 75208 &\ \ 198948 &\ \ 9853 \\
%she\/... &\ \ 55684 &\ \ 32600 &\ \ 28997 &\ \ 92891 &\ \ 2841 \\
%it\/... &\ \ 14109 &\ \ 8378 &\ \ 9007 &\ \ 37870 &\ \ 6216 \\\hline
%\end{tabular}\\{\footnotesize https://www.english-corpora.org/iweb/}
%\end{center}
%
%\begin{enumerate}\small\itemsep=0ex
%\item the object named \Q{player}, which has been created by \E{the game} and contains everything \E{it knows} about the player character
%\item \E{the team} is sure that \E{it knows} how to catch up
%\item Now when \E{the control software} reads G28 in a line then \E{it knows} to send the machine to this position
%\item Yeah, or someone would put a weighted dummy sitting in \E{the seat} so \E{it thinks} there is always a passenger
%\item as \E{the flight controller} fights to hold the craft in what \E{it thinks} is the correct attitude
%\item \E{the company} said \E{it thinks} Whole Foods shares are undervalued
%\end{enumerate}
%
\renewcommand{\refname}{\normalsize References}
\small
\bibliography{biblio}

\end{document}