%!TEX TS-program = xelatex
%!TEX encoding = UTF-8 Unicode

\def \papersize {letterpaper}  %a4paper

\documentclass[12pt,\papersize]{extarticle}
% extarticle is like article but can handle 8pt, 9pt, 10pt, 11pt, 12pt, 14pt, 17pt, and 20pt text

\def \ititle {}
\def \isubtitle {}
\def \iauthor {}
\def \iemail{}
%for anonymous submisison
%\def \iauthor {}
%\def \iemail{}
%\date{}

\input{$HOME/Documents/submissions/preamble_steve_paper3}

\begin{document}

\setlength\footnotesep{1em}

\bibliographystyle{newapa} %apalike

%these two lines are for anonymous submission --- they remove author and date
%but don't forget to remove defs above as well --- otherwise it will be in the metadata
\author{}
\date{}


%\maketitle
%\tableofcontents

% disables chapter, section and subsection numbering
\setcounter{secnumdepth}{-1} 

%\begin{abstract}
%\noindent
%\end{abstract}



\section{Teleology for Joint Action}
According to 
what Csibra and Gergely call `the principle of rational action',
%
\begin{quote}
`an action can be explained by a goal state if, and only if, it is seen as the most justifiable action towards that goal state that is available [to the agent] within the constraints of reality.'%
%
\footnote{
\citet[p.\ 255]{Csibra:1998cx}; cf.\ \citet{Csibra:2003jv}.
I think this principle needs refining \citep[pp.\ 6--7]{butterfill:2012_interacting}, but the refinements don't matter here.
A related but different `principle of efficiency' has been formulated by \citet[p.\ 1061]{Southgate:2008el}:
`goal attribution requires that agents expend the least possible amount of energy within their motor constraints to achieve a certain end.'
I'm not sure whether differences between this principle and the principle of rationality matter here.
}
%
\end{quote}
%
This principle can be used to predict an action given knowledge of a goal and the constraints.  
(It can also be used to infer goals from actions plus constraints, and constraints from actions plus goals \citep{Csibra:2003kp}.)
But it doesn't generally generate predictions where two or more agents are acting together.

To see why not, suppose that Ayesha and Bert are performing actions directed to the goal of moving the table from here to there. 
How could we use the principle of rational action to predict Ayesha's action?
We might try to treat Bert and his action as falling under `the constraints of reality'.
But this won't work if we suppose, further, that Ayesha's and Bert's actions are reciprocally coordinated. 
In that case, which is the `most justifiable' action for Ayesha depends on which is the `most justifiable' action for Bert; and conversely.
So we can't determine which action is the `most justifiable' for Ayesha and separately determine the same for Bert.
This is why the principle of rationality as formulated generates no prediction for joint actions which essentially involve reciprocal coordination.

But there is a natural way to generalise the principle of rationality for one or more agents:
%
\begin{quote}
One or more actions, <$a_1$, $a_2$, ...\ $a_n$>, can be explained by a goal state if, and only if, they are seen as the most justifiable action towards that goal state that are available to agents $S_1$, $S_2$ ...\ $S_n$
within the constraints of reality.
\end{quote}
%
This principle can be used to predict actions given knowledge of a distributive goal, the constraints and the fact that the agents are cooperative with respect to the distributive goal.
(Knowledge that the agents are cooperative with respect to the goal is necessary to use the principle for predicting their actions: if the agents were competing, the principle would generate incorrect predictions.)
	%But what does `cooperative' mean here?


What's the difference between the original principle of rationality and the generalised version?
Where the original concerns the fit between a single action to an outcome, the generalisation concerns the fit between a set of one or more actions and a goal.

Infants use the original principle in predicting individual actions. 
Will they also use the generalised principle in predicting joint actions?



\small
\bibliography{$HOME/endnote/phd_biblio}

\end{document}