%!TEX TS-program = xelatex
%!TEX encoding = UTF-8 Unicode

\def \papersize {letterpaper}  %a4paper

\documentclass[12pt,\papersize]{extarticle}
% extarticle is like article but can handle 8pt, 9pt, 10pt, 11pt, 12pt, 14pt, 17pt, and 20pt text

\def \ititle {}
\def \isubtitle {}
\def \iauthor {}
\def \iemail{}
%for anonymous submisison
%\def \iauthor {}
%\def \iemail{}
%\date{}

\input{$HOME/Documents/submissions/preamble_steve_paper3}

\begin{document}

\setlength\footnotesep{1em}

\bibliographystyle{newapa} %apalike

%these two lines are for anonymous submission --- they remove author and date
%but don't forget to remove defs above as well --- otherwise it will be in the metadata
\author{}
\date{}


%\maketitle
%\tableofcontents

% disables chapter, section and subsection numbering
\setcounter{secnumdepth}{-1} 

%\begin{abstract}
%\noindent
%\end{abstract}



\section{One-sided vs. reciprocal coordination}
In one-sided coordination, an agent attempts to coordinate her actions with those of another.
In reciprocal coordination, for herself and some particular other agents, an agent attempts to make it the case that their actions are coordinated.

%When an agent is engaged in reciprocal coordination, she might attempt to make her own actions more predictable so that the other agent can better coordinate with her.

Is the key difference we are after that in reciprocal coordination the agent takes the target with which she is trying coordinate to be an agent?
No.

I've lost track of exactly what we want.  
I think it's related to the difference between: (i) there being a goal which we each individually attempt to bring about, taking into account predictions about the other's actions; and (ii) there being a goal which we attempt to achieve by each considering what we need to do in order to bring it about.





\small
\bibliography{$HOME/endnote/phd_biblio}

\end{document}